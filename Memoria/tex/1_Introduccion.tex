\capitulo{1}{Introducción}

En este proyecto se va a desarrollar un programa que permita detectar objetos en el rango de visión de un láser de seguridad. En un principio estas medidas serán tomadas con el Hokuyo Safety Laser Scanner (UAM-05LP-T301). Este aparato es capaz de desarrollar tres áreas de detección dependiendo de la distancia a la que detecte un elemento (puede ser de 20, 10 y 5 metros). Aunque sea este láser con el que se va a realizar el proyecto, el principal objetivo es el de poder hacer que el programa pueda funcionar en diferentes tipos de láser con diferentes formas de obtención de datos.\\
La explicación básica del funcionamiento de este proyecto es la utilización de tres elementos bien diferenciados:
\begin{itemize}
    \item Láser: esta parte es la que se encarga de obtener los datos de su entorno. Estos datos serán enviados a través de una interfaz hacia la unidad de tratamiento de datos.
    \item Cable ethernet: es el soporte a través del cual se transportan tanto los mensajes enviado por el ordenador hacia el láser como las respuestas y los datos de lectura que este manda al PC demandante.
    \item PC: es la parte principal del proyecto, ya que es la encargada de permitir al usuario tanto introducir las áreas donde se necesita detectar los objetos, como las órdenes que el usuario desea transmitir al láser para recibir la información. Además también es la encargada de recibir y procesar las respuestas del láser para poder mostrar al usuario los resultados de la comparación de los datos y las coordenadas de las áreas introducidas. La utilización de este aparato es debida a que este proyecto sería destinado a ser introducido dentro de un AGV pero se necesita hacer visible al usuario a través de una pantalla.\\
\end{itemize}


Los AGV son una de las últimas innovaciones en el mundo de la industria. Son robots autónomos que sustituyen al trabajo humano en el sentido de transporte de objetos o mercancías entre las distintas secciones de las industrias. \\
Para la realización de sus propósitos estos aparatos, además de sus aparatos de guiado, necesitan aparatos que les permitan reconocer si en su camino existe algún obstáculo o si el objeto con el que trabaja (mercancía) se encuentra cerca de ellos. Para esto, los AGVs llevan incorporado, como norma general, un láser, el cual recoge un gran conjunto de datos sobre el entorno que lo rodea.\\
Actualmente, esta tecnología se basa en la configuración de áreas en las cuales el láser detecta las variaciones de puntos que se localizan dentro de estas áreas. Un inconveniente de esta configuración es la necesidad de que el área tenga uno de sus límites en el origen de coordenadas del gráfico trazado por las coordenadas obtenidas de las lecturas del láser por lo que para detectar si existe un objeto en un determinado límite situado en la zona media del gráfico antes mencionado pero no más cerca, se debe manejar y configurar más de un área.\\
En este proyecto se desarrollará un sistema con el que se pretende solventar el problema antes descrito, reduciendo el número de áreas a necesitar con lo que se conseguirá una mayor eficiencia en el análisis de entorno.\\
En este caso vamos a utilizar el Hokuyo Safety Laser Scanner (UAM-05LP-T301). Este aparato es capaz de desarrollar tres áreas de detección dependiendo de la distancia a la que detecte un elemento (puede ser de 20, 10 y 5 metros).\\