\capitulo{1}{Introducción}

En este proyecto se va a desarrollar un programa que permita detectar objetos en el rango de visión de un láser de seguridad. En un principio estas medidas serán tomadas con el Hokuyo Safety Laser Scanner (UAM-05LP-T301). Este aparato es capaz de desarrollar tres áreas de detección dependiendo de la distancia a la que detecte un elemento (puede ser de 20, 10 y 5 metros). Aunque sea este láser con el que se va a realizar el proyecto, el principal objetivo es el de poder hacer que el programa pueda funcionar en diferentes tipos de láser con diferentes formas de obtención de datos.\\
La explicación básica del funcionamiento de este proyecto es la utilización de tres elementos bien diferenciados:
\begin{itemize}
    \item Láser: esta parte es la que se encarga de obtener los datos de su entorno. Estos datos serán enviados a través de una interfaz hacia la unidad de tratamiento de datos.
    \item Cable ethernet: es el soporte a través del cual se transportan tanto los mensajes enviado por el ordenador hacia el láser como las respuestas y los datos de lectura que este manda al PC demandante.
    \item PC: es la parte principal del proyecto, ya que es la encargada de permitir al usuario tanto introducir las áreas donde se necesita detectar los objetos, como las órdenes que el usuario desea transmitir al láser para recibir la información. Además también es la encargada de recibir y procesar las respuestas del láser para poder mostrar al usuario los resultados de la comparación de los datos y las coordenadas de las áreas introducidas. La utilización de este aparato es debida a que este proyecto serái destinado a ser introducido dentro de un AGV pero se necesita hacer visible al usuario a través de una pantalla.\\
\end{itemize}
Debido a los programas utilizados el sistema operativo escogido para desarrollar todo este proyecto es Windows 10 debido a que es un sistema de gran difusión y a que es sencillo encontrar programas con los que desarrollar y probar el código además de comunicarse con el láser.\\
