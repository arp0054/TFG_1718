\capitulo{1}{Introducción}

En este proyecto se va a desarrollar un programa que permita detectar objetos en el rango de visión de un láser de seguridad. En un principio estas medidas serán tomadas con el Hokuyo Safety Laser Scanner (UAM-05LP-T301). Este aparato es capaz de desarrollar tres áreas de detección dependiendo de la distancia a la que detecte un elemento (puede ser de 20, 10 y 5 metros). Aunque sea este láser con el que se va a realizar el proyecto, el principal objetivo es el de poder hacer que el programa pueda funcionar en diferentes tipos de láser con diferentes formas de obtención de datos.\\
La explicación básica del funcionamiento de este proyecto es la utilización de tres elementos bien diferenciados:
\begin{itemize}
    \item Láser: esta parte es la que se encarga de obtener los datos de su entorno. Estos datos serán enviados a través de una interfaz (en nuestro caso un cable ethernet) hacia la unidad de tratamiento de datos.
    \item Unidad de tratamiento: es la parte fundamental del  proyecto. en este caso es un dispositivo MTX‐GTW. Este dispositivos es el encargado de ejecutar el código del programa e interactuar con el Láser, es decir, es la parte encargada de hacer operativo el proyecto.
    \item PC: esta parte es la encargada de permitirnos la creación del código así como la inclusión de las áreas en las que se desea detectar los objetos. Esta opción de introducción de áreas es realizada debido a que el proyecto se realiza exclusivamente con estas partes, ya que en su objetivo final sería el de introducirlo en un vehículo autoguiado o AGV.\\
\end{itemize}
Este proyecto se va a desarrollar en Un sistema operativo en base Linux, concretamente en un sistema Ubuntu en su version 16.04 LTS. Esto es debido a que ciertos componentes no son compatibles con sistemas  Windows.\\
