\capitulo{1}{Introducción}

Los AGV \cite{AGVdes} son una de las últimas innovaciones en el mundo de la industria. Son robots autónomos que sustituyen al trabajo humano en el sentido de transporte de objetos o mercancías entre las distintas secciones de las industrias. \\
\\
Para la realización de sus propósitos estos autómatas, además de sus sistemas de guiado, necesitan estructuras que les permitan reconocer si en su camino existe algún obstáculo o si el objeto con el que trabaja (mercancía) se encuentra cerca de ellos. Para esto, los AGVs llevan incorporado, como norma general, un láser de seguridad, el cual recoge un gran conjunto de datos sobre el entorno que lo rodea.\\
\\
Actualmente, esta tecnología se basa en la configuración de áreas en las cuales el láser detecta las variaciones de puntos que se localizan dentro de estas, representado estas variaciones de forma booleana a través de una alarma. Los láseres de seguridad poseen un número limitado de estas áreas, también denominadas campos de seguridad, las cuales, como bien se ha dicho antes, solo ofrecen información a cerca de si un objeto es detectado o no en un determinado campo.\\
\\
En este proyecto se desarrollará un sistema con el que se pretende solventar los problemas anteriormente descritos, ya que permitirá un número de áreas prácticamente ilimitado y la posibilidad de conocer de forma exacta la posición de los objetos que el láser detecte en su entorno, lo cual no se ha hecho hasta ahora con este tipo de láseres.\\
\\
