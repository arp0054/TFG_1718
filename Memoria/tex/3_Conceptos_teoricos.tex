\capitulo{3}{Conceptos teóricos}

En este proyecto se deben de emplear una gran cantidad de  conocimientos adquiridos a lo largo de las asignaturas del grado. En esta sección se describirán cuales son esos conceptos además de la forma en la que se aplican en este proyecto.

\section{Sistemas distribuidos}

Se denomina sistema distribuido a aquel sistema en el que cada equipo se conecta con el resto de los integrantes del sistema para poder compartir sus recursos además de poder utilizar los recursos que comparten el resto de los equipos conectados para realizar procesos para los cuales se necesitan una cantidad de recursos que la que el usuario de cada equipo posee de forma local. Otra forma de definir este concepto es la ejecución de procesos desde un equipo en otro cuyas características se adapten mejor a las del objetivo que se desea alcanzar.\\
En el caso de este proyecto, la forma en la que se emplea el concepto de sistema distribuido es la que se adapta a la última de las definiciones antes mencionadas. Esto se debe a que las características (tanto software como hardware) que posee el sistema MTX‐GTW son las idóneas para desarrollar este proyecto debido a que los sistemas sobre los que se va a utilizar este proyecto poseen características idénticas o muy similares a las de este dispositivo. En este caso, el equipo con el cual se desarrolla el código que forma la parte funcional del proyecto, se conecta vía ethernet con el dispositivo antes mencionado, el cual se conecta con el láser para interactuar con él, realizar las operaciones necesarias y devolver al equipo los resultados de esas operaciones.

\section{Minería de datos}

La minería de datos el la ciencia a través de la cual se analiza un gran conjunto de datos para poder descubrir características, patrones o comportamientos ocultos a primera vista.\\
Cuando el programa desarrollado se comunica con el láser, este debe recibir las tramas que el láser crea al analizar su área de observación, las cuales deben de ser tratadas para separar la información importante de las cabeceras y demás datos que el láser usa para interactuar con su aplicación. Una vez hemos conseguido extraer ese tipo de datos, hemos de descartar aquellos datos correspondientes a las zonas donde el láser no detecta ningún objeto, quedándonos solo con aquellas lecturas con valor significativo para la función principal del programa. Tras esto, se analizarán los datos para ver si coinciden con las áreas sobre las que se desea saber si hay objetos.

\section{Referencias}

Las referencias se incluyen en el texto usando cite \cite{wiki:latex}. Para citar webs, artículos o libros \cite{koza92}.


\section{Imágenes}

Se pueden incluir imágenes con los comandos standard de \LaTeX, pero esta plantilla dispone de comandos propios como por ejemplo el siguiente:

\imagen{escudoInfor}{Autómata para una expresión vacía}



\section{Listas de items}

Existen tres posibilidades:

\begin{itemize}
	\item primer item.
	\item segundo item.
\end{itemize}

\begin{enumerate}
	\item primer item.
	\item segundo item.
\end{enumerate}

\begin{description}
	\item[Primer item] más información sobre el primer item.
	\item[Segundo item] más información sobre el segundo item.
\end{description}
	
\begin{itemize}
\item 
\end{itemize}

\section{Tablas}

Igualmente se pueden usar los comandos específicos de \LaTeX o bien usar alguno de los comandos de la plantilla.

\tablaSmall{Herramientas y tecnologías utilizadas en cada parte del proyecto}{l c c c c}{herramientasportipodeuso}
{ \multicolumn{1}{l}{Herramientas} & App AngularJS & API REST & BD & Memoria \\}{ 
OSELAS Toolchain & X & & &\\
C++ & & & X &\\
Eclipse & & & X &\\
Git + GitHub & X & X & X & X\\
Mik\TeX{} & & & & X\\
\TeX{}Maker & & & & X\\
} 
