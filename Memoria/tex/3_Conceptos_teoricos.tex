\capitulo{3}{Conceptos teóricos}

En este proyecto se deben de emplear una gran cantidad de  conocimientos adquiridos a lo largo de las asignaturas del grado. En esta sección se describirán cuales son esos conceptos además de la forma en la que se aplican en este proyecto.

\section{Sistemas distribuidos}

Se denomina sistema distribuido a aquel sistema en el que cada equipo se conecta con el resto de los integrantes del sistema para poder compartir sus recursos además de poder utilizar los recursos que comparten el resto de los equipos conectados para realizar procesos para los cuales se necesitan una cantidad de recursos que la que el usuario de cada equipo posee de forma local. Otra forma de definir este concepto es la ejecución de procesos desde un equipo en otro cuyas características se adapten mejor a las del objetivo que se desea alcanzar.

\subsection{Uso en el proyecto}

En este caso, la forma en la que se emplea el concepto de sistema distribuido es la que se adapta a la última de las definiciones antes mencionadas. Esto se debe a que las características (tanto software como hardware) que posee el sistema MTX‐GTW son las idóneas para desarrollar este proyecto debido a que los sistemas sobre los que se va a utilizar este proyecto poseen características idénticas o muy similares a las de este dispositivo. En este caso, el equipo con el cual se desarrolla el código que forma la aprte funcional del proyecto, se conecta vía ethernet con el dispositivo antes mencionado, el cual se conecta con el láser para interactuar con él, realizar las operaciones necesarias y devolver al equipo los resultados de esas operaciones.


\section{Referencias}

Las referencias se incluyen en el texto usando cite \cite{wiki:latex}. Para citar webs, artículos o libros \cite{koza92}.


\section{Imágenes}

Se pueden incluir imágenes con los comandos standard de \LaTeX, pero esta plantilla dispone de comandos propios como por ejemplo el siguiente:

\imagen{escudoInfor}{Autómata para una expresión vacía}



\section{Listas de items}

Existen tres posibilidades:

\begin{itemize}
	\item primer item.
	\item segundo item.
\end{itemize}

\begin{enumerate}
	\item primer item.
	\item segundo item.
\end{enumerate}

\begin{description}
	\item[Primer item] más información sobre el primer item.
	\item[Segundo item] más información sobre el segundo item.
\end{description}
	
\begin{itemize}
\item 
\end{itemize}

\section{Tablas}

Igualmente se pueden usar los comandos específicos de \LaTeX o bien usar alguno de los comandos de la plantilla.

\tablaSmall{Herramientas y tecnologías utilizadas en cada parte del proyecto}{l c c c c}{herramientasportipodeuso}
{ \multicolumn{1}{l}{Herramientas} & App AngularJS & API REST & BD & Memoria \\}{ 
HTML5 & X & & &\\
CSS3 & X & & &\\
BOOTSTRAP & X & & &\\
JavaScript & X & & &\\
AngularJS & X & & &\\
Bower & X & & &\\
PHP & & X & &\\
Karma + Jasmine & X & & &\\
Slim framework & & X & &\\
Idiorm & & X & &\\
Composer & & X & &\\
JSON & X & X & &\\
PhpStorm & X & X & &\\
MySQL & & & X &\\
PhpMyAdmin & & & X &\\
Git + BitBucket & X & X & X & X\\
Mik\TeX{} & & & & X\\
\TeX{}Maker & & & & X\\
Astah & & & & X\\
Balsamiq Mockups & X & & &\\
VersionOne & X & X & X & X\\
} 
