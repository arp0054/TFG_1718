\capitulo{6}{Trabajos relacionados}

Los AGVs existen desde 1953, cuando su inventor diseñó un remolque que transportaba comestibles hasta un almacén recorriendo el camino marcado cpn una cable de acero el cual tiraba de él hasta su destino.\\
\\
Tras muchos años de innovación se han ido mejorando las estructuras de estos robots así como los elementos incorporados para su funcionamiento. Uno de los aspectos en los aspectos en los que más ha mejorado es en los sistemas de guiado. En la actualidad existen infinidad de sistemas de guiado para estos sistemas y uno de los más utilizados es el guiado por láser.\\
\\
Este sistema de guiado se basa en la detección de placas reflectantes con las que el AGV es capaz de orientarse para recorrer su camino gracias al análisis de los datos recibidos del láser e identificar dichas placas.\\
\imagen{laserAGV}{AGV guiado por láser}
Otra de las aplicaciones en las cuales se está aplicando la tecnología láser es en la automovilística donde se está empezando a aplicar para sistemas de frenado de seguridad en cuanto a la medida de distancias.\\
\imagen{distanciaSeg}{Medida de la distancia de seguridad vía láser}

Un proyecto no oficial es el desarrollado por el usuario de la plataforma GitHub iliasam. Este trabajo es en realidad mucho más mecánico, ya que se basa en la creación de un PCB en el cual se va a implementar un sistema de rotación y se va a conectar in lidar, el cual irá realizando lecturas para crear un mapa de puntos del entono detectado.\\
\\
Para poder obtener más información a cerca de este proyecto hay que visitar la URl: \url{https://github.com/iliasam/OpenSimpleLidar}