\apendice{Documentación técnica de programación}

\section{Introducción}
En este apartado se mostrarán todas las herramientas que necesita un programador para poder utilizar y trabajar en este proyecto. Como primer paso, antes de cualquier herramienta, se necesita acceso al proyecto.\\
\\
Este proyecto es accesible a través de la URL: \url{https://github.com/arp0054/TFG_ObjDetectorViaLaser}\\

\section{Estructura de directorios}
Debido a las diferentes formas de implementación que este proyecto ha tenido a lo largo de su desarrollo, se pueden observar una gran variedad de directorios:
\begin{itemize}
	\item DataSheets: en este directorio se encuentran los documentos técnicos y el protocolo de comunicación utilizado por el láser, los cuales han servido para obtener una gradn cantidad de datos que han ayudado al desarrollo del proyecto (dirección IP y número de puerto por defecto, formas de encriptación y desencriptación de los datos...).
	\item Diagramas: en este caso el nombre es muy descriptivo, en este directorio se encuentran los diagramas de clases y de secuencia, tanto los iniciales, desarrollados para las primeras fases del proyecto, como los finales. Estos diagramas se distinguen por estar separados en los subdirectorios "Iniciales" y "Finales".
	\item Dispositivo: también perteneciente a las fases iniciales, en él aparecen el manual de usuario del MTX-GTW, con detalladas explicaciones a cerca de sus necesidades eléctricas así como todas las aplicaciones que el hardware de este dispositivo es capaz de realizar.
	\item Instalación: en este caso, aparecen tanto imágenes como documentos a cerca de como instalar las librerías las cuales permitían la ejecución remota en el sistema descrito en el punto anterior así como otras utilidades necesarias para el desarrollo del proyecto en su primera versión (útiles para su uso con C$++$)
	\item Memoria: como su mismo nombre indica, este es el directorio en el que se guardan todos los archivos e imágenes que conforman la memoria de este proyecto, así como su sección de anexos.
	\item Python: en este directorio se alamcenan todos los archivos del sistema codificados en el lenguaje de programación con el mismo nombre, los cuales forman la versión actual y funcional del proyecto. Hay dos archivos con el mismo nombre " objectDetectorViaLaser" aunque con distinta extensión, dependiendo si es un archivo para usar con Jupyter Notebook (extensión ".ipynb"), el cual se utilizó primero para desarrollo y después solo para alguna prueba, o para utilización en Spyder(extensión ".py"). De la utilización de estos archivos aparece la generación automática de los dos subdirectorios pertenecientes a esta parte del repositorio.
	\item TFG\_ 1718: de nuevo un directorio de las primeras versiones, ya que este alberga todas las clases, ya sean funcionales o librerias, que fueron usadas en las primeras versiones para codificar el sistema objetivo en el lenguaje de programación C$++$ .
	\item URG\_ USB\_ Driver\_ en: este directorio almacena simplemente el driver utilizado para la conexión con el láser vía USB.
\end{itemize}
Como se ha podido observar,algunos de los directorios pertenecen a las primeras versiones. Esto viene dado por la motivación ta expuesta en el punto 7 de la memoria, donde una de las 
\section{Manual del programador}
En esta sección se explica como se configura el entorno de trabajo para este proyecto.
\subsection{Python}
Como primer paso para esta configuración se debe descargar el lenguaje de programación que se va a emplear, en este caso Python. Para este proyecto no se especifica ninguna versión en especial pero para su desarrollo se ha empleado la versión número 3.6.3.\\
Para  su descarga visitar: \url{https://www.python.org/downloads/release/python-363/}\\
\\
Después de descargar e instalar este lenguaje, el cual tiene incorporado el manejador de paquetes \textit{pip}, se descargan los módulos o paquetes que se emplean sobre todo en la representación gráfica con los siguientes comandos:
\begin{itemize}
	\item pip install matplotlib.
	\item pip install numpy
\end{itemize}
Estas bibliotecas son las que permiten preparar y representar los datos a mostrar de forma gráfica tras haber pasado por el proceso de análisis y traducción.\\

\subsection{IDE}
Para poder modificar o crear nuevo código en este proyecto se necesita un entorno de desarrollo integrado o IDE. Aunque Python ya tiene el suyo propio, se ha optado por el uso de otros debido a que son más amigables y algunos de ellos se han empleado a lo largo del grado. Estos entornos son: Jupyter Notebook y Spyder.\\
\\
Para descargar ambos entornos se debe instalar la aplicación de Anaconda Navigator, la cual incorpora a ambos. Para descargarlo hay que visitar la página web: \url{https://anaconda.org/anaconda/anaconda-navigator} \\
\\
La principal diferencia entre ambos es que Jupyter utiliza una interfaz web, mientras que Spyder posee su propia interfaz.\\
\imagen{interfazJupyter}{Interfaz gráfica de Jupyter}
\imagen{interfazSpyder}{Interfaz gráfica de Spyder}

\subsection{git}
Este es el sistema escogido para el control de versiones del proyecto. Al no estar incluido por defecto en Windows, se debe instalar para poder utilizarlo, al igual que el resto de programas utilizados. Para poder descargarlo se ha de acceder a la pagina cuya URL es: \url{https://git-scm.com/}\\
Dependiendo de las preferencias del programador se pueden instalar diferentes formas de instalación (consola de bash, de Windows..).

\subsection{GitKraken}
Este cliente de Git es el escogido para el mantenimiento de versiones en el repositorio, es decir, el cliente con el que se subirán cada uno de los cambios realizados al repositorio y con el  que se descargarán las últimas versiones del proyecto desde cualquier dispositivo empleado para el desarrollo de cualquiera de las partes del proyecto. Para descargarlo se debe acceder a: \url{https://www.gitkraken.com/download}\\
\\
Una vez descargado, hemos de acceder a GitHub y buscar el botón \textit{Clone or download} y copiar la URL que aparece tras presionarlo (\url{https://github.com/arp0054/TFG_ObjDetectorViaLaser.git}). Una vez copiada, se realizan los siguientes pasos en GitKraken: $boton con el icono de directorio \rightarrow CLONE \rightarrow Clone with URL \rightarrow se selecciona el directorio donde se van a guardar los archivos del directorio en el PC \rightarrow se pega la URL previamente copiada \rightarrow Clone repo! $\\
\\
Una vez completados esos pasos se habrá descargado el repositorio por lo que se puede usar como cualquier otro proyecto.\\
\imagen{interfazGitKraken}{Interfaz gráfica de GitKraken}

\section{Compilación, instalación y ejecución del proyecto}
Si se han cumplimentado todos los pasos anteriores en cuanto a instalaciones del entorno (Python y sus librerías) el sistema es capaz de compilar y ejecutar el software de este proceso (ya que no tiene proceso de instalación como tal).\\
\\
Para este proceso de ejecución se puede realizar de tres formas, atendiendo a los 3 entornos posibles:
\begin{enumerate}
	\item por consola de comandos (CMD en Windows): para este proceso se ha de abrir el programa \textit{Símbolo del sistema} (instalado por defecto en el sistema operativo), utilizar el comando \textit{cd} para colocarnos en la carpeta donde se encuentran todas las clases de Python y escribir el comando \textbf{objectDetectorViaLaser.py}
	\item Por Jupyter Notebook: se debe buscar el archivo objectDetectorViaLaser.ípynb desde la interfaz de Jupyter, la cual es similar a un explorador de archivos, y de ahi presionar el botón de ejecución tantas veces como bloques de código posea el archivo.
	\item Por Spyder: se debe de buscar el archivo desde el explorador de archivos, similar a cualquier otro programa, y ejecutarlo una única vez ya que en este caso no existe separación. 
\end{enumerate}
Ya que son equipotenciales, se deja a decisión del programador la forma en la que se ejecuta el proceso, ya que l compilación es automática en todos ellos.
\section{Pruebas del sistema}
