\apendice{Plan de Proyecto Software}

\section{Introducción}
En este apartado se especificará la planificación seguida en cuanto a tiempo se refiere para realizar cada una de las tareas que han conformado este proyecto, permitiendo gracias a esta planificación cumplir con los objetivos que se han establecido para el sistema desarrollado.\\
Debido a que gran parte de este proyecto se ha basado en la investigación (ya que, como se ha explicado en otros puntos, se desconocían algunos de los procedimientos a realizar), la planificación temporal no ha podido ser tan estricta como debería ser en un proyecto cuya organización se basa en la metodología ágil conocida como SCRUM como es este caso. Otro aspecto que ha impedido seguir esa planificación ha sido la aparición de una gran cantidad de contratiempos surgidos durante cada una de las fases de este proyecto, como se describirá más adelante.\\
En cuanto al estudio de viabilidad, la parte económica ha sido también un aspecto difícil de calcular ya que se ha necesitado probar una serie de implementaciones, tanto hardware como software, hasta dar con la idónea para cumplir los requisitos del proyecto.
 
\section{Planificación temporal}
Como bien se ha descrito en la introducción, en este apartado se especificará el desarrollo del proyecto descrito a partir de los periodos de tiempo que han llevado a desarrollar cada parte.
\section{Estudio de viabilidad}

\subsection{Viabilidad económica}

\subsection{Viabilidad legal}


