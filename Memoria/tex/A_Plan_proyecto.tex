\apendice{Plan de Proyecto Software}

\section{Introducción}
En este apartado se especificará la planificación seguida en cuanto a tiempo se refiere para realizar cada una de las tareas que han conformado este proyecto, permitiendo gracias a esta planificación cumplir con los objetivos que se han establecido para el sistema desarrollado.\\
Debido a que gran parte de este proyecto se ha basado en la investigación (ya que, como se ha explicado en otros puntos, se desconocían algunos de los procedimientos a realizar), la planificación temporal no ha podido ser tan estricta como debería ser en un proyecto cuya organización se basa en la metodología ágil conocida como SCRUM como es este caso. Otro aspecto que ha impedido seguir esa planificación ha sido la aparición de una gran cantidad de contratiempos surgidos durante cada una de las fases de este proyecto, como se describirá más adelante.\\
En cuanto al estudio de viabilidad, la parte económica ha sido también un aspecto difícil de calcular ya que se ha necesitado probar una serie de implementaciones, tanto hardware como software, hasta dar con la idónea para cumplir los requisitos del proyecto.\\
 
\section{Planificación temporal}
Como bien se ha descrito en la introducción, en este apartado se especificará el desarrollo del proyecto descrito a partir de los periodos de tiempo que han llevado a desarrollar cada parte, o como se conoce en terminología SCRUM, cada uno de los sprints.\\
\subsection{Sprint 0: 05/10/2017 - 22/11/2017}
Este es el sprint inicial, donde se explica y establece los conceptos sobre los que se va a trabajar y los elementos que se van a emplear para ello. Se plante la realización de este trabajo basandose en el sistema operativo Ubuntu, en su versión 14.04 LTS, para desarrollar el código usando la aplicación conocida como Eclipse en su versión 4.8 (conocida como Photon) y el leguaje de programación C$++$. Además se plantearon las librerías a utilizar para este proceso y la utilización del dispositivo MTX-GTW para la ejecución remota del sistema en este dispositivo.\\
\subsubsection{Tareas}
\begin{itemize}
	\item Creación de la máquina virtual con el sistema operativo antes mencionado.
	\item Configuración de Eclipse con las correspondientes librerías.
	\item Preparación de los diagramas que servirán como guía para el desarrollo del sistema.
	\item Búsqueda de cables de alimentación para cada uno de los elementos.
	\item Comienzo de investigación sobre la realización de ejecuciones remotas.
	\item Creación del repositorio.
\end{itemize}
\subsubsection{Backlog}
En este sprint inicial se plantea cada uno de los aspectos del proyecto (requisitos, objetivos...) así como el planteamiento de la organización con el tutor en cuanto a reuniones que se iban a realizar. Además se empezó a organizar la estructura del programa con la realización de los diagramas de clases.\\
\subsubsection{Investigación}
Se comienza la investigación a cerca de las ejecuciones remotas de programas para poder realizarlo con el MTX-GTW. Tambien se comienza a investigar a cerca del manejo de sockets y lecturas de puertos utilizando C$++$ para poder recoger los datos del láser y enviar mensajes por este mismo puerto para comunicarse con el láser (con la dificultad de que esta escucha también será remota).\\

\subsection{Sprint 1: 22/11/2017 - 16/01/2018}
En este sprint se plantean los cambios a realizar en el diagrama creado en el anterior sprint, se plantea el uso de \LaTeX para la realización de la memoria del proyecto y la creación del diagrama de secuencia para mejor comprensión del funcionamiento del sistema. Se comprueba también el estado de la máquina virtual y se detecta la falta de algunas librerías necesarias. \\
\subsubsection{Tareas}
\begin{itemize}
	\item Creación del diagrama de secuencia.
	\item Mejora del diagrama de clases según la revisión realizada.
	\item Solucionar la ausencia de librerías antes mencionado.
	\item Realización de pruebas de conexión con el MTX-GTW.
	\item Continuación de búsqueda de material hardware.
\end{itemize}
\subsubsection{Backlog}
En este sprint se termina de crear y modificar los diagramas que servirán de base a la posterior creación del código del sistema, se pretende configurar Eclipse con las librerías que faltaban del sprint anterior y se implementa el programa que imprima por pantalla "Hola mundo" realizando esta ejecución en el dispositivo remoto. Mientras se realizan estos procesos se prosigue con la búsqueda del cable de alimentación del láser ya que el cable del MTX-GTW ya ha sido encontrado.\\
\subsubsection{Investigación}
En este sprint se realizan varias tareas de investigación. En primer lugar, se deben encontrar las librerías que se necesitan para incluirlas en la configuración del proyecto en Eclipse. Esta tarea resulta costosa debido a que algunas de las librerías que se habían acordado utilizar estaban desactualizadas y no se podían encontrar y se desconocía la organización de ficheros y el funcionamiento de las nuevas versiones con lo que se hacia complicado la configuración para algunas de las tareas futuras.\\
La otra tarea de investigación realizada fue la búsqueda de los diferentes cables de alimentación y esto es debido a la diferencia entre ambos ya que el dispositivo MTX-GTW necesita ser alimentado con una corriente alterna de 12 V. Debido a que, en la actualidad, la mayoría de los cargadores de los diferentes aparatos electrónicos cotidianos no suministran más de 7 voltios se necesitó fabricar un cargador a base de un transformador que suministrase la corriente deseada unido a dos cable los cuales irán conectados al cabezal del aparato debido a que tampoco posee un cabezal convencional. Por otro lado, el láser  se han utilizado dos cableados distintos ya que se debe conectar tanto a la red eléctrica para alimentarse como al dispositivo antes mencionado. Para la primera de las conexiones descritas se ha empleado un adaptador a corriente alterna ya que este láser necesita alimentarse con 24 V de corriente alterna. Para la segunda conexión se necesita un cable con cabezal VGA y el otro extremos sin cabezal para poder conectar solo los cables que sean necesarios ya que solo se necesitan algunos de los pines de esta conexión para el envío y recepción de datos.\\

\subsubsection{Pruebas}
En este sprint se comienzan a realizar las primeras pruebas para la conexión del PC con el MTW-GTW para configurar de forma correcta esta conexión con el fin de tenerla de forma definitiva como conexión de ambos sistemas en el proyecto final.\\

\subsection{Sprint 2: 16/01/2018 - 01/02/2018}
Una vez se han creado los diagramas para aclarar en funcionamiento del sistema final, se plantean las tareas para este nuevo sprint. En este caso, una vez se posee una versión estable del programa que realiza las pruebas se empieza a plantear el inicio del desarrollo del código del sistema, así como la creación de una cuenta en la web Trello (mencionada en puntos anteriores) para la creación del panel con las tareas a realizar con lo que se consigue una version del panel de tareas para la organización de cada sprint.\\
\subsubsection{Tareas}
\begin{itemize}
	\item Paso del proyecto de pruebas al repositorio para comenzar en el el desarrollo del sistema 
	\item Creación de cuenta y tablón en Trello.
	\item Creacion de las primeras clases.
\end{itemize}
\subsubsection{Backlog}
En este sprint se ha comenzado con el desarrollo del sistema ya en el repositorio para mantener un control de las versiones que se van realizando. Tambien se comienza a dejar constancia de cada una de las tareas que se han plantado en los sprint, dividiéndolas en diferentes categorias segun se encuentren pendientes de realizar, en curso o si ya han sido realizadas.
\subsubsection{Investigación}
Se retoma la investigación a cerca de los sockets en el lenguaje de C$++$ para plantear la conexión con el láser con el objetivo de organizar la estructura final de la comunicación lógica de los diferentes elementos físicos que componen el sistema.

\subsection{Sprint 3: 01/02/2018 - 20/05/2018}
Este es uno de los sprints más largos debido a que ha sido en el que han surgido la mayoria de problemas. En este punto, al no poder plantear nuevas tareas debido al bloqueo de las ya presentes del anterior sprint, las reuniones se fueron separando unas de otras en el tiempo. 
\subsubsection{Tareas}
\begin{itemize}
	\item Solución de los problemas de Ubuntu y Eclipse.
	\item Continuación de la creación de las distintas clases y procedimientos.
\end{itemize}
\subsubsection{Backlog}
En este Sprint, como se ha dicho anteriormente, surgen la mayor parte de las incidencias que impiden el avance ene el proyecto por un largo periodo de tiempo. Una vez en proceso de creación del sistema, Eclipse comenzó a dar errores en cuanto a la comprensión de comandos básicos del lenguaje C$++$. Junto con esta clase de problemas surgió el inconveniente del fallo completo de Eclipse, ya que en la máquina virtual donde se estaba realizando todo el proceso, la aplicación dejó de funcionar, permitiendo verse en pantalla únicamente la barra de tareas. Para solucionar este fallo se realizaron algunas investigaciones sin éxito.\\
\subsubsection{Investigación}
Las investigaciones realizadas en este periodo de tiempo fueron varias aunque todas con el mismo objetivo, solucionar fallos.\\
En un inicio, los fallos en cuanto al lenguaje de programación. Para solucionar este tipo de fallo se realizón un análisis exaustivo de la configuración completa  del proyecto, incluyendo partes que no se habían visto hasta ese momento, por si se podría haber producido algún efecto en cascada que hubiese desencadenado el error (dicho análisis fue realizado de forma conjunta con el profesor). Tras comprobar que ese no era el problema se comenzó una búsqueda por diversas páginas web para encontrar soluciones de otros usuarios que hubiesen tenido el mismo problema. Al realizar esta búsqueda, no solo se descubrió que el número de personas cone este error era muy reducido, sino que además se pudo observar que las soluciones que se les ofrecía para solucionarlo o no funcionaban o no eran útiles para el problema que se planteaba en nuestro caso. Como conclusión a este problema se decidió crear otro proyecto y dejar para más adelante la configuración realizada con anterioridad.\\
En cuanto al problema de la visualización de Eclipse, la situación era similar a la del anterior problema. En este caso el número de personas con el mismo problema era mayor pero la solución que se daba en la mayoría de los casos era la eliminación de los recursos temporales generados al arrancar Eclipse para reiniciarlo y que los crease de nuevo, intentando solventar así los fallos que hayan podido aparecer durante el arranque. Al no conseguir solucionarlo de esta manera, se optó por la opción más drástica, extraer los cambios ya realizados en el proyecto creado para evitar el primer problema, eliminar la máquina virtual y comenzar la instalación y configuración de aplicaciones desde el principio.\\
Este método no funcionó, por lo que se decidió tomar un camino alternativo para el proyecto.

\subsection{Sprint 4: 20/05/2018 - 13/06/2018}
Tras el periodo anterior de cambios de máquina virtual y reconfiguraciones se tomó la decisión de cambiar el proyecto. A partir de este sprint el proyecto sería realizado en Windows 10 (sistema anfitrión del PC) para evitar problemas con los servicios de virtualización. Debido a esa decisión se hacia incompatible la utilización del MTX-GTW, ya que este aparato es incompatible con Windows, por lo que la ejecución del sistema se realizaría en el PC. Este aspecto solucionó otro de los problemas anteriores, las librerías. Al no tener que realizar ninguna ejecución remota, el proyecto solo iba a necesitar utilizar las librerías que ya vienen por defecto en C$++$, sin necesidad de descargar ninguna otra. En este periodo se comienza las primeras interacciones con el láser
\subsubsection{Tareas}
\begin{itemize}
	\item Configuración del proyecto y desarrollo del código en Windows.
	\item Investigación sobre programas de manejo del láser.
	\item Cambios importantes en la memoria.
\end{itemize}
\subsubsection{Backlog}
En este periodo de tiempo se consigue la realización de gran parte del código que forma el sistema software ya que no se producen errores como los del anterior sprint. También se instala y empieza a investigar el funcionamiento del programa que se encuentra adjunto al láser en un CD presente en la misma caja del aparato. Junto con este programa, tambien se descarga e instala el programa RealTerm, el cual se usará para envir y recibir mensajes por el puerto USB, ya que el láser tiene la posibilidad de conectarse al PC por esta vía.
\subsubsection{Investigación}
En este aspecto se empieza a investigar la configuración del programa RealTerm antes mencionado, dejando para más adelante la programación de la parte restante del sistema, para centrarse en averiguar como conseguir la conexión con el láser además de conoces como realizar el intercambio de información.\\
Tambien se investiga la forma de configurar del programa del láser (UAM Proyect Designer) para poder ver el funcionamiento de este programa y la visualización por pantalla de las lecturas del láser, lo cual se podrá tomar como guía en la representación del resultado de la ejecución del sistema objetivo. Esto se consigue gracias a la configuración de los puertos USB de PC para que permitan instalar drivers no firmados (como son los del propio láser, los cuales tambien se tuvieron que buscar por internet para evitar la dependencia con el CD) con lo que se pudiese reconocer al láser para que el programa funcionase correctamente. Este buen comportamiento también se produjo ya que el tutor facilitó un fichero de configuración básica el cual fue enviado vía USB al propio láser, el cual lo conserva en memoria desde entonces.
\subsection{Sprint 5: 13/06/2018 - 19/11/2018}
En este sprint, cuya duración se debe a la aparición del último de los problemas de gran relevancia para el proyecto y la presencia del periodo de verano lo cual supone una mayor dificultad para las reuniones. En este sprint se analizan los resultados del ciclo anterior, observando las lecturas del láser por primera vez desde el inicio del proyecto. El problema antes mencionado surge al intentar crear el socket lógico en C$++$ tras conocer como crear los mensajes y provocar la comunicación entre los dos aparatos y conseguir asi implementar una de las partes más importantes del proyecto. No se consigue (entre el alumno y el profesor) que el socket conecte el PC con el láser con lo que se dedica este periodo a solucionar este problema
\subsubsection{Tareas}
\begin{itemize}
	\item Solucionar falta de comunicación con el láser desde el PC.
	\item Cambio de lenguaje del proyecto
\end{itemize}
\subsubsection{Backlog}
En este periodo se realizan algunas investigaciones con resultados nada satisfactorios y, tras no conseguir el bojetivo principal de este sprint, se decide cambiar de lenguaje. En este caso se pasa todo el código existente a Python. Este lenguaje, al ser más usado, posee un soporte mucho mayor, especialmente por parte de los usuarios que lo usan para desarrollar su código, por lo que la traducción del código se realiza de forma rápida y sencilla. Gracias a la utilización de este lenguaje se consigue configurar el socket y tenerlo plenamente operativo de forma muy sencilla, con lo que en el periodo final de este sprint se consigue solucionar el priblema inicial y provocar un grán avance.
\subsubsection{Investigación}
En este periodo la primera investigación, como ya se ha comentado anteriormente, es la solución del socket en el lenguaje de C$++$, sin resultado satisfactorio. A parte de esta, la investigación principal se centró en la creación de sockets en Python lo cual se consiguió de forma muy sencilla gracias a la ayuda proporcionada por la propia web del lenguaje y los usuarios que poseian este problema en diferentes foros web.
\subsubsection{Pruebas}
Se realizan las primeras pruebas de comunicación y obtención de respuestas por parte del láser que da como resultado la implementación del procedimiento que permite extraer la lectura y conservar esos datos en el sistema para su posterior análisis.

\subsection{Sprint 6: 19/11/2018 - 11/12/2018}

\subsubsection{Tareas}
\begin{itemize}
	\item 
\end{itemize}
\subsubsection{Backlog}

\subsubsection{Investigación}

\subsection{Sprint :  - }

\subsubsection{Tareas}
\begin{itemize}
	\item 
\end{itemize}
\subsubsection{Backlog}

\subsubsection{Investigación}

\section{Estudio de viabilidad}

\subsection{Viabilidad económica}

\subsection{Viabilidad legal}


