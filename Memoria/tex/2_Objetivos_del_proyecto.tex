\capitulo{2}{Objetivos del proyecto}

En este apartado se explican los distintos objetivos identificados en este proyecto, distinguiendo entre los objetivos generales, técnicos y funcionales del mismo.\\

\section{Objetivos generales}

Se tiene por objetivo crear un software capaz de realizar un análisis de los datos recibidos del láser para identificar la posición de los objetos de su entorno. Por lo tanto, los objetivos generales de este proyecto son:
\begin{itemize}
	\item Crear una conexión para la comunicación con el láser
	\item Intercambiar mensajes con el láser para obtener sus datos de lectura.
    \item Realizar un análisis de dichos datos para comprobar si existe o no un objeto en las áreas requeridas por el usuario y conocer la posición del mismo.
\end{itemize} 

\section{Objetivos funcionales}
Los objetivos funcionales estimados para este proyecto son:
\begin{itemize}
	\item El usuario podrá escoger el número de zonas que desea analizar, siendo este número tan grande como desee.
	\item El usuario decidirá cada una de las coordenadas en las que se encuentran los límites de las áreas que pretende analizar.
	\item El usuario podrá ejecutar cuantas veces quiera el programa para poder observar diferentes lecturas.
	\item El usuario podrá observar una serie de mensajes "true" o "false" (en el orden de inserción de las áreas anteriormente establecidas) para comprobar en qué áreas se ha detectado un objeto y en cuales no.
	\item El usuario podrá observar de forma gráfica (más concretamente en un gráfico de puntos) los resultados del análisis y la posición de las áreas establecidas.
	\item El usuario será capaz de observar la distancia a la que se encuentran los objetos detectados en el análisis.
\end{itemize}

\section{Objetivos técnicos}

Los objetivos técnicos planteados para este proyecto son:
\begin{itemize}
	\item Aprender el uso de Python para la creación del código del proyecto, a través de todas sus funcionalidades previstas con el objetivo de la lectura de datos de las lecturas del láser.
	\item Usar librerías de Python no empleadas hasta el momento en ninguna asignatura del grado como binascii o re, las cuales han servido para poder manejar los datos recibidos y desarrollar el comportamiento deseado.
	\item Encontrar todos los elementos hardware necesarios para la creación del sistema necesario para el proyecto.
	\item Crear un proyecto el cual implementa la funcionalidad que se requiere en los objetivos del proyecto, es decir, que realice la lectura, tratamiento y análisis de los datos del láser.
	\item Utilizar un sistema de control de versiones como es Git, más concretamente uno de sus servicios centrales más utilizados conocido como GitHub. Esta herramienta es útil en el caso de producirse algún error grave en alguna de las partes del proyecto ya que se puede restaurar la versión previa (la cual se considera estable).
\end{itemize}

