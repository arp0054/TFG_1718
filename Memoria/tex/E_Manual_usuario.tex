\apendice{Documentación de usuario}

\section{Introducción}
En este apartado se verán todos los aspectos necesarios para poder instalar y usar el software de forma correcta.

\section{Requisitos de usuarios}
Para que un usuario pueda usar este sistema, debe poseer antes el hardware necesario para ello:
\begin{itemize}
	\item Un ordenador: en el cual se va a ejecutar el software. Este necesita tener como sistema operativo Windows 10, ya sea como sistema operativo principal o el un sistema virtualizado, aunque se recomienda más la primera de las opciones debido a que se evitan posibles fallos a causa de los servicios ofrecidos por las máquinas virtuales.
	\item Un láser: con el que se van a realizar las lecturas de entorno. En este caso se recomienda el mismo utilizado para el desarrollo (Hokuyo Safety Laser Scanner  o UAM-05LP-T301) ya que el programa ha sido diseñado y se encuentra configurado para él. Si este láser se cambiase por otro se tendrían que cambiar algunos de los parámetros más importantes del sistema (mensajes utilizados, forma de traducción, configuración del socket...).
	\item Cable ethernet: empleado para la comunicación de los dos elementos anteriormente nombrados. En este caso, la elección del cable se deja a elección del usuario, aunque se recomienda el creado por la empresa desarrolladora del láser o uno con características similares.
	\item Transformador eléctrico: que suministre 24 V de corriente, voltaje requerido por el láser para poder funcionar correctamente. Para esto se utilizará también un par de empalmes de cables para conectar los cables del láser (los cuales no poseen cabezal convencional) a los cables del transformador.
\end{itemize}
\section{Instalación}

\section{Manual del usuario}


