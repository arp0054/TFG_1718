\capitulo{4}{Técnicas y herramientas}

Esta parte de la memoria tiene como objetivo presentar las técnicas metodológicas y las herramientas de desarrollo que se han utilizado para llevar a cabo el proyecto. Si se han estudiado diferentes alternativas de metodologías, herramientas, bibliotecas se puede hacer un resumen de los aspectos más destacados de cada alternativa, incluyendo comparativas entre las distintas opciones y una justificación de las elecciones realizadas. \\
No se pretende que este apartado se convierta en un capítulo de un libro dedicado a cada una de las alternativas, sino comentar los aspectos más destacados de cada opción, con un repaso somero a los fundamentos esenciales y referencias bibliográficas para que el lector pueda ampliar su conocimiento sobre el tema.


\section{Desarrollo de la memoria}

En este aspecto se ha querido emplear LaTeX en contraposición a otros programas de propósito similar como pueden ser Microsoft Word o Open Office Writer ya que esta aplicación nos permite una mayor cantidad de posibles acciones a realizar en el documento crear. \\
Por poner un ejemplo de estas ventajas antes mencionadas, la presentación del documento puede quedar más limpia ya que el usuario puede manejar el espacio de cada hoja a su antojo y poder disponer de todo él ya que se puede variar los márgenes de las páginas de una forma sencilla (introducción de un comando), lo que en los otros programas antes mencionados resultaría demasiado complicado a demás de ser potencialmente peligroso debido a que, gracias a las acciones automáticas que poseen, podrían desajustar todo el contenido de la hoja en sí haciendo que el usuario tenga que ajustar cada uno de los elementos que antes poseía. LaTeX permite saltarse los márgenes establecidos, por ejemplo, para insertar una imagen. en cuanto a los encabezados, pies de página y numeración de página también se puede realizar con un simple comando. \\
Con todo esto y con la característica de que permite guardar el archivo directamente en PDF sin realizar ninguna conversión (aunque para ello necesite crear algún archivo debido a la compilación) es por la que se ha escogido para este 

\section{Desarrollo del código}

Debido a que este entorno de desarrollo ya se ha empleado en algunas asignaturas del grado, lo cual lo hace más amigable y facil de usar, se ha decidido emplear Jupyter Notebook para el desarrollo del código. Además de ser compatible con el sistema operativo, este entorno tiene la ventaja de poder obtener la conexión con el láser, y por lo tanto el intercambio de mensajes y datos con el mismo, sin necesidad de instalar ningún complemento para poder desarrollar el código sin dificultades añadidas.
Este entorno, además de las ventajas antes descritas, permite desarrollar y ejecutar (y por lo tanto testar) partes del código por separado. Por este motivo, es más fácil comprobar el correcto funcionamiento del código desarrollado antes de realizar otras partes del código siendo después más difícil rectificar dicho error.

\section{Planteamiento de las tareas}

Para organizar las actividades a realizar se ha escogido la aplicación Trello. Esta aplicación se ha escogido debido a su facilidad de uso (debido a que su interfaz de usuario es muy intuitiva), a demás de otros muchos aspectos como el hecho de que a un tablero (unidades organizativas en las que se gestionan las tareas de los diferentes proyectos que se puede gestionar desde una misma cuenta de usuario) se puede acceder más de un usuario. Esta característica permite que tanto el alumno como el profesor pueden acceder al mismo tablero para gestionar las tareas a realizar.

\section{Metodología de gestión y herramientas asociadas}

Para este proyecto se ha decidido utilizar la metodología de SCRUM. Esta metodología esta basada en entregas incrementales pero funcionales. Para la realización de esta metodología, perteneciente a las denominadas metodologías ágiles, se va a utilizar la aplicación ZenHub. Esta aplicación se puede emplear para presentar los sprints y se puede planificar la fecha de comienzo y final cada una de las tareas.

\section{Patrones de diseño empleados}

Los patrones de diseño son herramientas reutilizables empleadas para resolver problemas que resultan comunes a la hora tanto de desarrollar el software como el diseño de las interacciones e interfaces empleadas para que el usuario pueda emplear el sistema.\\
En el desarrollo de el sistema en el que se basa el proyecto se han utilizado los siguientes patrones:
\begin{itemize}
    \item Plantilla: para la realización de las clases \textit{Láser} e \textit{Interfaz} ya que aunque cada interfaz y cada láser tienen ciertas funciones específicas hay una parte de estas que son comunes por lo que, para evitar tener que duplicar código y mejorar la eficiencia de estas partes del sistema.
    \item Singleton: en este proyecto se va a crear la clase \textit{Procesador} como un Singleton ya que solo se debería de crear una instancia de este tipo ya solo es necesario esta instancia para el funcionamiento del  sistema. Con la utilización de este patrón nos aseguramos de solo tener una instancia de la clase antes mencionada lo que mejora el control del flujo de datos a demás de la eficiencia del sistema.\\
\end{itemize}

\section{Herramienta de manejo de láser}

Aunque el manejo de este elemento del proyecto se va a realizar a través de funciones presentes en Eclipse y el código implementado para el proyecto, se va a utilizar una aplicación instalada vía CD llamada UAM Proyect Designer. Esta aplicación, cuando el equipo en el que se instala está conectado al láser, permite observar los datos relacionados con la nube de puntos creada por este aparato al realizar sus operaciones de lectura de su entorno.\\
En este caso esta herramienta será utilizada para la comprobación de que el código desarrollado para alcanzar los objetos del proyecto funciona de forma correcta comparando los resultados obtenidos por el código y la aplicación, los cuales deberían de coincidir.\\
Como bien se ha dicho al principio de esta sección, la aplicación se ha instalado utilizando el CD presente en el paquete en el que el láser se encuentra almacenado. 

\section{C++}

El lenguaje de programación con este nombre es aquel desarrollado en 1979 para la ampliación del lenguaje existente conocido como C. Se creó con el objetivo de orientar ese lenguaje a los objetos, con lo que se ampliaba las posibilidades de uso del mismo.\\
Este lenguaje ha sido el escogido para desarrollar todo el código del programa. Este lenguaje suele ser el más indicado debido a que suelen emplear algunas librerías que actúan con el equipo a bajo nivel con lo que sería tremendamente recomendable ya que sería necesario acceder directamente a los pines del RS232 del MTX-GTW a través del cable del ethernet.

\section{Cableado}

Para que el proyecto pueda ser desarrollado y ejecutado para la comprobación de su correcto funcionamiento es necesario suministrar corriente eléctrica a los componentes. Para ello, debido a que los dos componentes principales necesitan distintos tipos de corriente eléctrica, se han tenido que buscar dos elementos diferentes:
\begin{itemize}
	\item MTX-GTW: este elemento necesita ser alimentado con una corriente alterna de 12 V. Debido a que, en la actualidad, la mayoría de los cargadores de los diferentes aparatos electrónicos cotidianos no suministran más de 7 voltios se necesitó fabricar un cargador a base de un transformador que suministrase la corriente deseada unido a dos cable los cuales irán conectados al cabezal del aparato debido a que tampoco posee un cabezal convencional.\\
	\item Láser: para este aparato se han utilizado dos cableados distintos ya que se debe conectar tanto a la red eléctrica para alimentarse como al dispositivo antes mencionado. Para la primera de las conexiones descritas se ha empleado un adaptador a corriente alterna ya que este láser necesita alimentarse con 24 V de corriente alterna. Para la segunda conexión se necesita un cable con cabezal VGA y el otro extremos sin cabezal para poder conectar solo los cables que sean necesarios. El primero de los cables ha sido buscado y encontrado por el alumno mientras que el segundo ha sido suministrado por el tutor.
\end{itemize}