\capitulo{5}{Aspectos relevantes del desarrollo del proyecto}

\section{Definición del trabajo como proyecto de investigación}

Un aspecto que hay que destacar de este proyecto es que su enfoque, además de ser el de crear un sistema funcional que cumpla los requisitos establecidos en puntos anteriores, es el de crear una simulación del entorno en el que posteriormente se implementará este sistema, buscando los elementos tanto hardware como software (programas, lenguajes, librerías, recursos...)  necesarios para que un sistema ya implantado en su entrono de trabajo pueda utilizar el sistema desarrollado para realizar algunas de sus tareas. Como resumen, se podría decir que es un trabajo que pretende desarrollar un sistema para la investigación de la implementación de un nuevo sistema de detección de objetos.\\

\section{Conocimientos utilizados aprendidos en el grado}

Algunos de los conocimientos que se han empleado en este proyecto se han adquirido a lo largo de las diferentes asignaturas de los 4 cursos del grado.\\

\subsection{Codificación en Python}
El conocimiento principal que ha permitido desarrollar el código funcional de este proyecto es el conocimiento a cerca de la creación de código en lenguaje Python. Gracias a algunas asignaturas como son Sistemas Inteligentes ($1^{er}$ semestre del $3^{er}$ curso), Algoritmia (2º semestre del $3^{er}$ curso), Nuevas Tecnologías y empresa y Minería de Datos (ambas del 2º semestre del 4º curso).\\
Gracias a haber recibido clases de estas asignaturas se han podido conocer y aprender a manejar algunas de las librerías que se han empleado en la creación de este sistema software, como pueden ser Numpy, Matplotlib o Math, además de los conocimientos a cerca de las características principales de este lenguaje de programación.\\

\subsection{Conocimiento sobre sockets}
Aunque en este sistema se ha tratado este asunto con poco detalle, el manejo de sockets también es un concepto que se ha sabido emplear gracias a alguna asignatura del grado. Un ejemplo de estas asignaturas es Sistemas Distribuidos (2º semestre del 4º curso).
Gracias a los conocimientos adquiridos en esta asignatura se ha podido configurar y gestionar el comportamiento del socket lógico creado para la comunicación entre el láser y el ordenador.\\

\subsection{Conocimiento a cerca de la gestión del proyecto}
Este aspecto es uno de los más relevantes ya que es la parte del proyecto la cual ayuda a gestionar el tiempo que se usa para llegar al objetivo marcado por los requisitos del proyecto. Los conceptos de la definición, aspectos y formas de aplicación de las metodologías ágiles, SCRUM en el caso de este proyecto, han sido adquiridos gracias a haber recibido las clases de la asignatura de Gestión de Proyectos ($1^{er}$ semestre del $3^{er}$ curso).\\

\section{Conocimientos externos a lo aprendido en el grado}

También existen conocimientos aprendidos gracias al desarrollo de este proyecto los cuales no se han adquirido a lo largo del grado.\\
En cuanto a este proyecto, se ha de destacar uno solo de estos conocimientos, la creación de los mensajes. Como ya se ha explicado en apartados anteriores, para comunicarse con el láser, el PC debe enviar una serie muy concreta de comandos, los cuales deben de temen una codificación específica. Esta codificación se basa en que algunas de las partes del mensaje (inici y final del mensaje y CRC) se deben enviar codificadas como Bytes mientras que el resto deben ser enviados como código ASCII. Ninguna de estas codificaciones ha sido impartida en ninguna de las asignaturas que imparten código en Python mencionadas en el punto anterior.\\
Estos conocimientos tienen la ventaja de que, al ser conceptos sobre uno de los  lenguajes de programación más usados, es sencillo encontrar mucha información al respecto por internet, por lo que se pueden encontrar diferentes formas de obtener el mismo  resultado (utilizando diferentes librerias o creando métodos con las librerías que están por defecto en el propio lenguaje) con lo que el programador puede escoger la que le parezca más idónea para su propósito.