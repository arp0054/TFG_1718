\capitulo{5}{Aspectos relevantes del desarrollo del proyecto}

Al igual que muchas otras partes de este documento, este aparatado debe de ser dividido en dos partes diferenciadas:

\section{Parte Software}

En cuanto a este aspecto del proyecto, la primera parte que se realizó fue la descarga e instalación de la máquina virtual mencionada en el documento, lo cual no supuso dificultad alguna debido a que es un proceso repetido varias veces a lo largo de los 4 cursos del grado.\\
El siguiente paso, el cual tuvo algo más de complicación fue el de la integración del entorno de desarrollo MTX-GTW. Las indicaciones para la realización de este proceso fueron obtenidas de la URL \url{http://mtxm2m.com/wiki/guia-software-mtx-gtw-mtx-gateway/}. Aunque ciertas partes de este tutorial de instalación no ha sido posible su instalación, no eran aspectos relevantes para el desarrollo de la actividad, no fue un impedimento y se pudo desarrollar el proyecto con normalidad.\\

\section{Parte Hardware}

En la parte hardware también existe un manual que ha sido utilizado para comprender y saber emplear l dispositivo empleado para almacenar y ejecutar el código que utiliza el programa. Se puede ver en la URL \url{ http://mtxm2m.com/wiki/mtx-gtw_}\\
Otro aspecto relevante es la búsqueda realizada para encontrar los materiales y eqipos necesarios para poder manejar el MTX-GTW, debido a que se trata de una tecnología que no suele ser utilizada para ser manejada desde un PC, por lo que se ha debido de simular su entorno habitual de trabajo.\\