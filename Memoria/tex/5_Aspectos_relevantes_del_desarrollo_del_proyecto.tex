\capitulo{5}{Aspectos relevantes del desarrollo del proyecto}

Al igual que muchas otras partes de este documento, este aparatado debe de ser dividido en dos partes diferenciadas:

\section{Parte Software}

En cuanto a este aspecto del proyecto, la primera parte que se realizó fue la descarga e instalación de la máquina virtual mencionada en el documento, lo cual no supuso dificultad alguna debido a que es un proceso repetido varias veces a lo largo de los 4 cursos del grado.\\

El siguiente paso, el cual tuvo algo más de complicación fue el de la integración del entorno de desarrollo MTX-GTW. Las indicaciones para la realización de este proceso fueron obtenidas de la URL \url{http://mtxm2m.com/wiki/guia-software-mtx-gtw-mtx-gateway/}. Aunque ciertas partes de este tutorial de instalación no ha sido posible su instalación, no eran aspectos relevantes para el desarrollo de la actividad, no fue un impedimento y se pudo desarrollar el proyecto con normalidad.\\

Otra parte relevante del proyecto fue la integración de los elementos software necesarios para la comunicación del equipo con el láser. En un primer momento se intentó usar los procedimientos ofertados por el proveedor ROS (URL: \url{http://wiki.ros.org/es}). ROS se encarga de proveer a sus usuarios las librerías necesarias para el desarrollo de software orientado a la creación de programas específicos para robots.Este sistema causo problemas a la hora de su instalación por lo que se intentaron buscar otras medidas para realizar la comunicación antes descrita. En esta búsqueda se descubrió (a través del acceso a la información del láser en su página oficial) la existencia de una librería específica para  C++ (lenguaje con el que se está desarrollando el proyecto) con la que realizar las relaciones requeridas.

\section{Parte Hardware}

En la parte hardware también existe un manual que ha sido utilizado para comprender y saber emplear l dispositivo empleado para almacenar y ejecutar el código que utiliza el programa. Se puede ver en la URL \url{ http://mtxm2m.com/wiki/mtx-gtw_}\\

Otro aspecto relevante es la búsqueda realizada para encontrar los materiales y equipos necesarios para poder manejar el MTX-GTW, debido a que se trata de una tecnología que no suele ser utilizada para ser manejada desde un PC, por lo que se ha debido de simular su entorno habitual de trabajo.\\

Como primer elemento del que se necesito realizar una búsqueda fue el elemento cargador. El MTX-GTW no posee un cargador al uso, ya que por norma general los dispositivos alimentados a base  de batería poseen cargadores con un conector que proporciona una tensión de entre 5 y 7V. Por el contrario, para alimentar a nuestro dispositivo se necesita un cargador sin conector, es decir, cuya terminación sean dos cables independientes (ya que deben ser conectados al cabezal propioo del dispositivo) y que abastezca de una tensión continua de 12V. Este elemento se consiguió encontrar a base de usar un transformador al que se le conectaron dos cables para enlazarlo al MTX-GTW.\\
El otro aspecto para el cual se ha necesitado buscar un elemento hardware es la comunicación entre el MTX-GTW y el láser. Para este aspecto, esta búsqueda es debida a que, aunque el primero de los elementos posea un conector VGA, el segundo no posee un cabezal al uso, ya que posee un conector de 12 pines independientes colocados en fila.