\apendice{Especificación de Requisitos}

\section{Introducción}
En este apartado se presentan las funcionalidades básicas que debe cumplir el sistema desarrollado para este proyecto. En este caso se van a diferenciar dos tipos: los objetivos generales, los cuales son de obligado cumplimiento, y el resto de requisitos que se mostrarán en un catálogo debido a que  no es necesario que se cumplan todos.

\section{Objetivos generales}
Para el sistema software desarrollado en este proyecto, se fijan los siguientes objetivos generales:
\begin{itemize}
	\item Establecer una conexión con el láser.
	\item Intercambiar mensajes para obtener los datos de lectura.
	\item Implementar un analizador de dichos datos.
	\item Implementar una representación gráfica para la visualización del análisis.
\end{itemize}

\section{Catalogo de requisitos}
Para la realización de este catálogo se diferenciará entre dos tipos de requisitos: los requisitos funcionales y los requisitos no funcionales.
\subsection{Requisitos funcionales}

\subsection{Requisitos no funcionales}
\begin{itemize}
	\item El sistema ha de detectar de la forma más precisa posible la distancia a la que se encuentra cada uno de los puntos detectados, ya se encuentren dentro o fuera de los campos de área establecidos.
	\item El software necesita estar conectado al láser para poder realizar la comunicación.
	\item Al ser una primera versión de este proyecto, es decir, al poder ser un sistema extensible y escalable, se ha codificado de forma modular.
\end{itemize}
\section{Especificación de requisitos}


