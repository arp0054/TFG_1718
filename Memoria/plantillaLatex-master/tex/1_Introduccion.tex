\capitulo{1}{Introducción}

En este proyecto se va a desarrollar un programa que permita detectar objetos en el rango de visión de un láser de seguridad. En un principio estas medidas serán tomadas con el Hokuyo Safety Laser Scanner (UAM-05LP-T301). Este aparato es capaz de desarrollar tres áreas de detección dependiendo de la distancia a la que detecte un elemento (puede ser de 20, 10 y 5 metros). Aunque sea este láser con el que se va a realizar el proyecto, el principal objetivo es el de poder hacer que el programa pueda funcionar en diferentes tipos de láser con diferentes formas de obtención de datos.
La explicación básica del funcionamiento de este proyecto es la utilización de tres elementos bien diferenciados:
\begin{itemize}
    \item Láser: esta parte es la que se encarga de obtener los datos de su entorno. Estos datos serán enviados a través de una interfaz (en nuestro caso u cable ethernet) hacia la unidad de tratamiento de datos.
    \item Unidad de tratamiento: esta es la parte más importante del proyecto, esto es debido a que es la parte que gestiona las inmensas cantidades de datos recibidas del láser, procesarlos y manejarlos. La otra parte por la que esta parte es esencial es la de que también se comunica con la parte encargada de establecer las áreas en las que el programa debe de comprobar si hay un objeto con los datos recibidos del láser en un momento determinado. Esta es la parte con la que va a interactuar el sujeto que utilice la aplicación.
    \item Gestor de áreas: Es otra de las partes del proyecto con la que le usuario va a actuar sobre el programa ya que es aquella en la que va a definir las áreas donde quiere saber si existe un objeto o no.
\end{itemize}

Este proyecto se va a desarrollar en Un sistema operativo en base Linux, concretamente en un sistema Ubuntu en su version 16.04. Esto es debido a que ciertos componentes no son compatibles con sistemas  Windows.
