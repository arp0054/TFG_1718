\capitulo{3}{Conceptos teóricos}

\section{Funcionamiento de las cajas negras}

Para este proyecto se nos asignó el dispositivo MTX-GTW \cite{gtx-mtx}, el cual es un dispositivo con linux embebido que actúa como caja negra en los distintos vehículos de guiado autónomos. Este dispositivo, no solo actúa como caja negra, sino que realiza otras tareas más técnicas como puede ser resolver problemas de tráfico, etc.

Los ficheros con los que tenemos que trabajar en el proyecto, son generados a través de scripts de Python en los cuales almacenan la trama que es recibida a través del conector CAN 2.0B \cite{CAN} del dispositivo.
\imagen{busCAN}{Diagrama de un BUS CAN. Fuente: wikimedia.org.}
Estas líneas, o tramas tienen el siguiente formato:

30 09:58:04:534923 03C 1B00060004110000


Donde 30 sería el día, 09:58:04:534923 sería la hora hasta en microsegundos, 03C es el identificador del contenedor de la trama de variables, en el cual, tendremos que especificar en el XML como dispondrá dicha trama, y que variables va a contener.

El contenido de la trama se almacena en 1B00060004110000 el cual almacena una o varias variables.

\imagen{tramas}{Ejemplo de una pequeña cantidad de tramas que analiza la herramienta del proyecto.}
Hay que especificar en esta memoria, que cada variable es única para contenedor, y que tendrán por tanto identificadores únicos.
A la hora de escoger lenguaje y plataforma del dispositivo, la manera óptima de realizar este proyecto hubiera sido tratar con una base de datos este tipo de variables y que corriera todo el servidor en el dispositivo, pero esto no fue posible por 2 motivos:
\begin{itemize}
	\item No se podían instalar programas nuevos puesto que era un sistema Linux muy básico, y estos programas tenían que venir instalados por parte del fabricante, el cual mediante cadenas de compilación se tenía que adaptar a ese sistema operativo concreto.
	
	\item Al ser un elemento de un proceso industrial, con capacidades de procesamiento muy limitadas, con 458 Mhz, y apenas 128 MB de RAM, con otros procesos activos, como habíamos mencionado, corriendo en segundo plano, la opción de correr en la parte del servidor no era válida, puesto que era más que probable que el dispositivo se congelase, y el procesamiento de extracción de variables sería extremadamente lento.
	
\end{itemize}
\section{Pasos del proyecto}

En el ciclo de vida del proyecto, se intento dotar la menor carga posible al datalogger. Para ello, el grueso de las operaciones de cálculo, como sería calcular todas las bibliotecas y operaciones de filtros y búsqueda de datos, se encuentra en la parte del cliente.

\newpage


\tablaSmall{Dispositivo en el que son ejecutadas cada proceso de la vida del proyecto}{l c c }{dispositivosejecucion}
{ \multicolumn{1}{l}{Dispositivo} & Datalogger & Navegador web cliente\\}{ 
Almacenamiento en ficheros de tramas CAN.  & X & \\
Envío de contenido de página. & X &\\
Mapeo de los ficheros del dispositivo. Servidor PHP. & X &\\
Envio de los ficheros del dispositivo. & X &\\
Aplicación del XML a las tramas de los ficheros. & & X\\
Almacenamiento de las tramas en la biblioteca. & & X\\
Parseo del XML en javascript. & & X\\
Utilización y graficado de dichas bibliotecas. & & X\\
Filtros y búsquedas. & & X\\
Exportación e importación de vistas. & & X\\
} 

\newpage
\section{Conceptos de lenguajes}
\subsection{XML}
\imagen{xml-logo}{Logo XML. Fuente: Freepik.es.}

Es un lenguaje cuyo objetivo es diseñar lenguajes de marcado. Este fue creado para poder ofrecer la información de la forma más estructurada, abstracta posible, con la opción de poder ser reutilizable.\cite{XML}

Es por eso que ha sido utilizado en nuestro proyecto para especificar la disposición de las tramas.


Escogí XML frente a otros lenguajes para establecer la configuración de las variables puesto que es mucho más visible y estructurado de cara al usuario a la hora de especificar una configuración, ya que este XML tiene el objetivo de ser continuamente adaptado y modificado.


En nuestra estructura sigue el siguiente formato, donde puede haber distintas tramas por configuración, y distintas viables por trama:


\imagen{xml_disposicion}{XML con la estructura de la disposición de tramas}

Como se aprecia, el lenguaje es muy fácil de entender y es por eso que fue seleccionado como fichero de configuración a la hora de la lectura dinámica de variables.

Se estudió añadir XSD \cite{XSD}(documento esquema), que es un convenio fijado para la estructura de un XML. Se decidió al final no emplearlo puesto que en cierto modo limitaba la libertad que XML nos daba, y no era necesario para el funcionamiento o marcar cierto orden en la estructura, puesto el XML ya de por si, considerábamos que estaba suficientemente estructurado.

\subsection{JSON}
\imagen{json-logo}{Logo JSON. Fuente: https://qph.ec.quoracdn.net/.}
JSON es un formato de lenguaje independiente, el cual trabaja como notación de objetos en JavaScript.

Fue seleccionado para importar y exportar vistas, puesto que estando trabajando con JavaScript es perfecto para importar y exportar objetos.

Como JSON es formato de texto ligero, permite pequeñas modificaciones en dichas vistas siempre y cuando sean compatibles con las herramientas. \cite{JSON}

Por lo tanto, fue seleccionado, por encima de la alternativa de XML en esas tareas concretas.

\subsection{CSS}
\imagen{animacionCSS}{Animación con CSS. Gracias a CSS podemos realizar imagenes como esta.}
Conocido en español como hoja de estilos, es un lenguaje utilizado para definir los estilos y propiedades respecto a la estructuración de los distintos elementos que se encuentran en una página. \cite{CSS}

Surge con la idea de separar la estructuración de dichos elementos fuera del HTML.

La versión aplicada en este proyecto es CSS3.

\subsection{PHP}

PHP es un lenguaje de programación libre, el cual está publicado con una licencia libre similar a la GNU, llamada PHP, con algunas restricciones.  Este lenguaje tiene el objetivo de ser empleado en el servidor (parte del servidor), siendo de los primeros lenguajes (fue creado en el 1995 por Rasmus Lerdorf) que se podían incorporar directamente contenido dinámico a través de un documento HTML.\cite{PHP5.5}

El código que contiene, es procesado por el servidor, gracias al módulo de procesador de PHP, donde este genera el contenido dinámico antes mencionado.
Además, PHP provee al programador de una interfaz con línea de comandos, para ayudar a este en posibles tareas como pueden ser aplicaciones gráficas.
\imagen{php55}{Logo PHP 5.5. Es la versión utilizada en este proyecto. Fuente: https://community.dynatrace.com}


La potencia de PHP reside en que puede usado en casi todos los sistemas operativos y plataformas, lo cual le convierte en un lenguaje universal y gratuito.

PHP es criticado normalmente por ciertos aspectos en la actualidad. Entre estos motivos son:
\begin{itemize}
	\item La lentitud de un script de PHP respecto a un lenguaje de bajo nivel, cuyo efecto puede ser reducido gracias a usar en archivos y en memoria técnicas de cache.
	\item En versiones anteriores a la 7, las cuales en la actualidad (2017), apenas nadie usa, las variables ofrecen un tipado flojo, esto conlleva que a la hora de utilizar la mayoría de los IDEs no nos puedan ofrecer la asistencia común que nos ofrecen otros IDES.

\end{itemize}
Es un lenguaje difícil de ofuscar. Esto conlleva que cualquiera puede entender nuestro código, aunque nosotros no queramos, con lo que esto conlleva.

Nosotros, como ya hemos mencionado en la memoria de este proyecto, nada más usamos PHP para que mapee 2 directorios, que es donde se encuentran los ficheros. En el cual, con una llamada síncrona, pasamos el directorio listado.

\subsection{Javascript}

Conocido por sus siglas JS, es un lenguaje de programación interpretado, débilmente tipado y orientado a objetos. Está basado en el tipificado ECMASCRIPT como dialecto
Es utilizado normalmente en el lado del cliente, (no tenemos en cuenta la versión de JavaScript para servidor, SSJS), el cual es interpretado por los distintos navegadores web, concediendo la posibilidad de crear web dinámicas.\cite{JS}

Actualmente todos los navegadores modernos (desde el 2012) soportan la versión de JavaScript ECMASCRIPT (5.1)\cite{ES}.
\imagen{js_ecma}{Evolución JS/ECMASCRIPT. Fuente: http://adrianmejia.com}

La sintaxis de este lenguaje es una mezcla de lenguajes, el cual es bastante similar a C, pero utiliza ciertos aspectos de Java, como pueden ser nombres y convenciones.

JavaScript suele ser normalmente confundido por Java, pero no tienen nada que ver como lenguajes, ya que son completamente diferentes.

JavaScript interactúa con el DOM de la página (document object model), en el cual están todos los objetos de la página, que como hemos dicho, con los que puede interactuando, aunque este es uno de los distintos usos utilizados.

Uno de los principales inconvenientes de JavaScript, es que programadores usan este lenguaje de forma inadecuada y con fines ilícitos, con relativa facilidad, por lo que es una de las vías de insertar código malicioso vía web.


Uno de los principales debates entre programadores \cite{JS_OOD}, es que hay algunos no consideran JavaScript como un lenguaje orientado a objetos. Yo considero este lenguaje como un lenguaje orientado a objetos, pero técnicamente y estrictamente no lo es, porque para que un lenguaje sea orientado a objetos tiene que cumplir las siguientes condiciones:
\begin{itemize}
	\item Tiene que soportar el polimorfismo, que si que lo cumple. Puesto que la mayoría de los lenguajes dinámicos lo hacen.

	\item Tiene que soportar la encapsulación, lo cual también lo cumple, puesto que una de las principales características de JavaScript es la facilidad de encapsulación que este ofrece.

	\item Sin embargo, el último atributo es la herencia, la cual técnicamente no la soporta, únicamente permite la herencia a través de prototipos, pero técnicamente esto está a expensas de la encapsulación. De hecho, en nuestro proyecto no hemos utilizado la herencia por esto mismo.

\end{itemize}

Aun así, JavaScript sigue siendo para mí un lenguaje orientado a objetos, puesto que podemos implementar un OOD (Diseño orientado a objetos) en JavaScript.

\subsection{HTML5}
\imagen{HTML5}{Logo HTML5. Fuente:http://oleblogs.com}

Es la quinta versión, y la más moderna versión utilizada en la actualidad, por el lenguaje de marcado HTML, este se utiliza para estructurar el contenido de las páginas web. \cite{HTML5}

Esta versión está regulada por W3C \cite{W3C}, el cual es un consorcio internacional cuyo objetivo es asegurar el crecimiento la web a largo plazo.

La sintaxis de este lenguaje consiste en etiquetar una serie de elementos, en forma de contenedor, con una serie de etiquetas, que pueden contener o no contenido propio, por el cual, el navegador web interpretará y mostrará al usuario.\cite{HTML}

HTML5 además incorpora códecs para mostrar contenidos multimedia como videos, de forma que se ha ido el mermando el uso de flash player en la web, el cual contenía muchas vulnerabilidades y sustituyendo por reproducción nativa HTML5.
También destaca la incorporación de etiquetas para manejar muchos datos, ciertas mejoras en los formularios para evitar el uso de JavaScript, visores  para fórmulas matemáticas (MATHML) , funcionalidades para arrastrar todo tipo de objetos como imágenes y visores de gráficos vectoriales (SVG).
