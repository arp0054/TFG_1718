\capitulo{6}{Trabajos relacionados}

En este apartado hablaremos de proyectos similares al realizado,  que existen actualmente en el mercado.

\section{Grafana}
Es el trabajo relacionado más famoso de todos, y es empleado para monitorizar todo tipo de datos. 

A diferencia de mi proyecto, grafana permite cargar directamente los datos desde una base de datos, lo que es muy útil en empresas que quieran mostrar datos a sus clientes.

Grafana es gratuita (open source)\cite{GRAFANA}, y está basada en cuadros de mandos fácilmente editables , que hacen de ésta una herramienta muy potente, donde el usuario selecciona que datos ver, y como los quiere ver; dentro de las diversas opciones que te permiten seleccionar.

A pesar de ello, los gráficos que ofrece a veces son bastante confusos, y no esta planteado para trabajar con cursores, a pesar de ser un proyecto que admite muchas funcionalidades adicionales, al ser código open source.

\section{Graphite}
Al igual que grafana, graphite es una herramienta open source que permite monitorizar y graficar datos de sistemas informáticos en tiempo real.
Destaca por la exactitud del tiempo real con la que ofrece sus datos.

Graphite funciona de la siguiente forma\cite{GRAPHITE}:
\begin{itemize}
	\item Un demonio en segundo plano, escucha los datos a tiempo real.
	\item Estos datos se guardan en una base datos.
	\item Una webapp llamado Django, renderiza los puntos en tiempo real.
\end{itemize} 

\section{ChartBlocks}

Este es un editor muy fácil de usar, que permite realizar gráficos con relativa facilidad.

A diferencia del mi proyecto, y de los dos anteriores, destaca por su facilidad de uso, aunque en uso, no es tan potente como los anteriores.\cite{CHARTBLOCKS}

Permite leer datos de distintas fuentes, incluso de datos en tiempo real y destaca la facilidad de su interfaz.



