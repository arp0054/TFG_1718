\apendice{Documentación técnica de programación}


\section{Introducción}
En este apartado, vamos a indicar los pasos de la instalación y configuración necesarios para ejecutar este proyecto.
La página del proyecto es : 
\url{https://github.com/rmb0033/analizadortramas}
\section{Estructura de directorios}
Dentro de la estructura de nuestro proyecto, la cual se encuentra alojada en:

\url{https://github.com/rmb0033/analizadortramas}

Podremos ver distintos directorios y archivos, donde se detallará su uso a continuación.

Estructura de directorios de la aplicación:

\subsection{Carpeta raiz.}

Se encuentran:
\begin{itemize}
	\item La imagen la universidad de Burgos.
	\item La clase PHP utilizada para mapear los ficheros.
	\item El archivo de lectura para el repositorio.
	\item El archivo XML utilizado para la lectura de variables.
	\item El archivo XML utilizado como pruebas para un único valor.
	\item La página HTML principal.


\end{itemize}

\subsection{Datalogger} 

Es la carpeta donde se encuentran los ficheros que tienen que ser analizados. Este directorio cuelga un subdirectorio, REPOS. El cual también contiene esos datos.  

Estos ficheros y directorios fueron proporcionados por el tutor como enunciado del proyecto.

\subsection{CSS} 
 
\begin{itemize}
	\item Es el directorio  donde se encuentran las hojas de estilo utilizadas en el proyecto. En la carpeta principal estarán los 2 spinners utilizados para la gráfica, los cuales han sido modificados y la hoja de estilos principal.

	\item Dentro de los directorios que contiene css, estarán “librerías”, la cual contiene hoja de estilos de terceros (bootstrap) y fonts que también contiene fuentes creadas por terceros.

\end{itemize}


\subsection{JS} 

Es el directorio donde se encuentran los archivos Javascript de la aplicación. En la carpeta principal se encuentra el fichero main.js, el cual es el fichero principal que llama a los demás archivos.
Dentro de JS tenemos dos directorios:
\begin{itemize}
	\item Vendor: En el cual se encuentran las librerías de terceros utilizadas en el proyecto.

	\item Modelos. La cual alberga todas las librerías creadas por el programador del proyecto.

\end{itemize}
Dentro de esas librerías se encuentran:
\begin{itemize}
	\item Behaviors: Se encuentran las clases encargadas de asignar comportamientos a los elementos de la página web y de la gráfica.

	\item Biblioteca:
Se encuentran las clases encargadas del almacenamiento de los datos de cada variable.
	\item CargadorFicheros: Se encuentran las clases encargadas tanto de cargar las lineas de los ficheros, como de cargar los XML \cite{XML}. El cargador de ficheros se encarga también de unir tanto las clases que se encuentran en biblioteca como en configuración. 

	\item Configuración: Se encuentran las clases encargadas de almacenar la información de la disposición de las variables, y para ello, de analizar el XML para obtener dicha disposición.
	
	\item Interfaz: Es donde se encuentra los archivos javascript encargados de gestionar la interfaz gráfica de la creación y gestión de vistas.

	\item Gráfica: En el directorio raíz se encuentran las clases de variables y gráficas, las cuales son las encargadas de configurar y crear tanto gráficas como variables. A su vez tiene otro directorio llamado procesamiento de datos: Se encuentran las clases encargadas de calcular los intervalos que cumplen una consulta y aplicarlos. ya sea a través de una búsqueda o a través de un filtrado.

\end{itemize}



\section{Manual del programador}
En el siguiente anexo, se explicará como dejar a punto el sistema para que se pueda ejecutar el servidor y a su vez nos permita trabajar en nuestro ordenador como herramienta de trabajo. Este manual de programador está adaptado para Windows, pero tiene semejanzas con los demás sistemas operativos.

\subsection{PhpStorm}
Se trata de nuestro IDE y por tanto nuestra herramienta principal de trabajo. PhpStorm \cite{PHPSTORM} ofrece a profesores y alumnos licencias gratuitas, por lo que no tendrá ningún coste para nosotros.
Para ello entramos en el siguiente link y descargamos la aplicación:

\url{https://www.jetbrains.com/phpstorm/download/#section=windows}

\imagen{manual1}{Pagina oficial de PHPStorm}

Una vez descargado, seguimos los pasos de la instalación:

\imagen{manual2}{Paso 1 de la instalación}
Para evitar incompatibilidades seleccionamos estas opciones:

\imagen{manual3}{Paso 2 de la instalación}

Por último instalamos:
\imagen{manual4}{Paso 3 de la instalación}
Antes de realizar ningún cambio vamos a proceder a instalar git.

\subsection{Git}
Git es un sistema de control de versiones, que es necesario para trabajar con nuestro repositorio \cite{GIT}. Para ello, como Windows no trae git por defecto tenemos que instalarlo.
\imagen{manual5}{Página de descarga de git}

Procedemos a instalar git:
\imagen{manual6}{Paso 1 de instalación de git}
\imagen{manual7}{Paso 2 de instalación de git}

Marcamos estas opciones:
\imagen{manual8}{Paso 3 de instalación de git}
\imagen{manual9}{Paso 4 de instalación de git}
Seguimos con la instalación: los demás pasos de la instalación deberán ser los establecidos por git por defecto.


\section{Compilación, instalación y ejecución del proyecto}

\imagen{manual10}{Primera imagen al abrir la aplicación}
Abrimos PHPStorm, por defecto no importaremos opciones. Aceptaremos las condiciones del contrato y entraremos en nuestra cuenta gratuita.
\imagen{manual11}{Carga de PHPStorm}
Una vez dentro, seleccionaremos la opción de:

“Check out from Version Control”


Aquí podremos clonar nuestro repositorio \cite{GITHUB} :
\url{https://github.com/rmb0033/analizadortramas.git}

\imagen{manual12}{Carga de repositorio en PHPStorm.}

Seleccionaremos la opción de clone. Y se nos bajará el repositorio.
\imagen{manual13}{Carga de repositorio en PHPStorm.}
Nos saldrá un cartel advirtiendo como este:
\imagen{manual14}{Advertencia de carga repositorio.}
Si no queremos/podemos realizar el clone. Siempre podremos descargar el repositorio y copiar el contenido de este dentro de un nuevo proyecto.

Tendremos que instalar el intrepete PHP. Para ello vamos al apartado de PHPStorm donde nos mandan instalar el interprete:
\imagen{manual15}{Selección interprete PHP}

Para ello necesitamos la versión 5.5 o 5.6 de PHP.

Accedememos a la siguiente página: 
\url{http://windows.php.net/download/}
y descargamos una release de PHP 5.6. 
Para este proyecto hemos usado la siguiente:
\url{http://windows.php.net/downloads/releases/php-5.6.30-nts-Win32-VC11-x86.zip}

Una vez descargado el zip, procedemos a descomprimirlo en una carpeta conocida.
\imagen{manual16}{Carpeta con contenido de PHP 5.6}

En PHPStorm, siguiendo con la última pantalla. Presionamos en las opciones de CLI Interpreter:
\imagen{manual26}{Selección del interprete CLI. Paso 1}

Pulsamos en el icono “+” y seleccionamos “Local Path”.
\imagen{manual17}{Selección del interprete CLI. Paso 2}

Recorremos la carpeta que hemos descomprimido de PHP hace unos instantes y seleccionamos el interprete (php-cgi-exe) y aceptamos.
\imagen{manual18}{Selección del interprete CLI. Paso 3}
Ahora el “proyect configuration” debería tener este aspecto:
\imagen{manual19}{Aspecto final configuración del proyecto PHPStorm.}

\subsection{Realizar cambios en el repositorio.}
Utilizaremos esta opción:
\imagen{manual20}{Vista general desde PHPStorm. VCS.}

Por último, si queremos subir los cambios a github utilizaremos un commit and push, siempre y cuando tengamos acceso a ese repositorio.
\imagen{manual21}{Progreso barra control de versiones}

Nos mandará loguearnos antes de hacer el push por primera vez en esta pantalla:
\imagen{manual22}{Pantalla de login de gitHub}

Por último, la aplicación nos devolverá un mensaje de éxito si se ha realizado correctamente el commit:
\imagen{manual23}{Mensaje de éxito en VCS}

Observamos los cambios si se han realizado con éxito:
\imagen{manual24}{Commit en github del nuevo commit realizado desde github}

\subsection{Para ejecutar el proyecto} 
Simplemente vamos al index.html. Seleccionamos de la barra superior el menú contextual “RUN” y seleccionamos el navegador con el que queremos abrir nuestro proyecto.
\imagen{manual25}{Commit en github del nuevo commit realizado desde github}

Por defecto utilizamos Chrome, que tiene que estar instalado en nuestro equipo para poder ser ejecutado nuestro proyecto.

De esta manera ya tendríamos nuestro equipo funcionando, y de la misma manera que tenemos preparado nuestro entorno de desarrollo podríamos ejecutar nuestro proyecto de forma completa.

\section{Pruebas del sistema}

Al tratarse una aplicación que básicamente está centrada en un eso en interfaz gráfica, hemos descartado las pruebas unitarias y nos hemos centrado en realizar bugtesting en la aplicación en ejecución.

Para ello, la aplicación ha sido probada en entornos reales de producción, para verificar el correcto funcionamiento de forma  manual, de principio a fin, donde en la última versión,la aplicación no ha producido errores, asegurándonos completamente del buen funcionamiento del proyecto dentro del dispositivo.

Se ha probado la aplicación tanto en navegadores con versiones actuales (1-7-2017) de Google Chrome, como Mozilla Firefox.


