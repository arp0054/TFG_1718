\capitulo{4}{Técnicas y herramientas}

\section{Técnicas de desarrollo}

\subsection{Metodología ágil}

Para el desarrollo de este proyecto se ha resuelto mediante metodología ágil.


Como resultado de los resultados de las antiguas metodologías, surgen nuevas soluciones en el tiempo, con las cuales se consigue un aumento en el desarrollo software basándose en dos principios:\cite{MAgil}

\begin{itemize}
	\item Postergar las decisiones.
	\item Planificación adaptativa, es la planificación que otorga a los miembros del equipo de desarrollo poder, para que puedan estar preparados para posibles modificaciones en el transcurso del proyecto, incluso la toma de decisiones mientras transcurre este último. 
\end{itemize}

Esto conlleva que al cliente se le vayan entregando versiones funcionales del proyecto de software, cada vez más completas, desde una versión básica hasta conseguir el resultado deseado por el cliente.

Estas nuevas versiones, cada vez más funcionales pueden ser cambios o modificaciones, que el equipo de desarrollo tendrá que hacer frente, de tal manera, que sin aumentar el coste se desarrolle un software flexible para tener la capacidad de aceptar.

En un caso práctico en nuestro proyecto, en un principio se optó por utilizar una librería totalmente distinta, que no tenía nada que ver con la entrega del proyecto. Gracias que en un principio los datos estuvieron desarrollados esperando posibles cambios, el código de extracción de datos apenas sufrió modificaciones.

Por tanto, podríamos resumir un proceso de metodología ágil, a aquel que cumpla las siguientes características:
\begin{itemize}
	\item Existan numerosas entregas de software funcional
	\item Comunicación dentro del equipo y con el cliente durante todo el proyecto
	\item Continuidad y motivación en el tiempo de desarrollo
	\item Estado del proyecto se mide por la funcionalidad de la aplicación deseada por el cliente.

\end{itemize}


\subsection{SCRUM}

\imagen{scrum}{Logo Scrum. Fuente: https://www.codercamps.com.}


Dentro de las distintas metodologías ágiles hemos escogido la metodología scrum, que nos ha permitido dirigir y administrar el desarrollo del software.

Como característica de metodología ágil, hemos ido realizado en el desarrollo del proyecto distintas versiones incrementales, en las cuales hemos añadido nuevas funcionalidades y cambiando/arreglando otras\cite{SCRUM}.

En primer lugar, se llevó a cabo una definición completa de requisitos con el cliente, en tal caso, el tutor del proyecto.

 Una vez establecido los requisitos tutor y alumno acordamos distintas fases en la ejecución del proyecto, marcando distintos tiempos en la ejecución de estos.

Semanalmente, iba informando del progreso del proyecto con el cliente, en este caso con el tutor, y le mostraba las distintas funcionalidades que había aplicado en cada iteración, en muchos casos, insatisfactorias, por lo que se procedía en la reunión a recoger las posibles modificaciones en dichas versiones, y modificar nuestras fases para adaptar estas modificaciones. 

La inspección final de la revisión era cuando el cliente daba el visto bueno a dichas iteraciones, dándome pie a trabajar en otros nuevos.

Uno de los pasos de scrum, es que se cuenta con un equipo con distintos roles. 

Al realizar este proyecto una sola persona, el reparto no fue equitativo y resultó que adquirí más de un rol. 

Al tratarse de un proyecto realizado por una sola persona, se trató al tutor como cliente (product owner), y yo adquirí las tareas de scrum máster y de equipo de desarrollo: anotando y preparando nuevas iteraciones, así como encargándome de su desarrollo integral.


Esta metodología ágil, denominada Scrum, como podéis ver, no puede considerarse rígida en su planteamiento, y seguramente otros alumnos hayan realizado esta metodología en sus proyectos con diferentes variantes, pero cumpliendo siempre con los principios y adaptando las necesidades del proyecto al equipo de personas que participan en la metodología SCRUM.
\newpage
\section{Herramientas de documentación}

\subsection{LaTeX}
\imagen{latex}{Captura del editor TexMaker que trabaja con Latex.}
Latex es un editor de texto que permite ser usado para realizar artículos científicos por su alta calidad tipográfica a partir de macros. \cite{LATEX}
En el proyecto hemos seleccionado esta opción pues el resultado era definitivamente mejor que en OpenOffice, aunque presentó alguna dificultad en su puesta en marcha.

Para este trabajo se ha utilizado el editor de latex TexMaker \cite{TEXMAKER},que es gratuito bajo la licencia GPL. 
 
Página oficial editor Latex: http://www.xm1math.net/texmaker/

\section{Herramientas de gestión}

\subsection{Sprintometer}
\imagen{sprintometer}{Captura de la página oficial de Sprintometer.}


Sprintometer es una herramienta de seguimiento de metodología ágil para proyectos Scrum y extreme programming.
Esta herramienta en principio fue creada para trabajar en proyectos ágiles con sus propios propósitos y ahora es una herramienta freeware.\cite{SPRINTOMETER}

Hemos usado esta herramienta para registrar los sprints, a su vez divididos en tareas para después comparar los tiempos estimados con los reales.

Página oficial: http://sprintometer.com/

\subsection{Git}
\imagen{git_logo}{Logo de Git.Fuente: git-for-windows.github.io}

Hemos utilizado git, como software de control de versiones\cite{GIT}. Este nos ha permitido controlar las distintas versiones de nuestro proyecto.
Su uso es gratuito y tiene una licencia GNU GPL V2.
\subsection{Github}


Se ha escogido esta plataforma de desarrollo colaborativo, ya que este proyecto es open source.

Su utilidad es para almacenar en la nube, gracias a git, el control de versiones de un producto\cite{GITHUB}.
\imagen{github}{Captura del estado del proyecto en github.}

Github no es siempre es gratuito, y tiene distintos tipos de usuarios. Nosotros usamos el modelo gratuito el cual nos obliga a que nuestros proyectos sean públicos.
Página:  http://github.com




\section{Herramientas de desarrollo}


\subsection{PHPStorm}
\imagen{phpstorm}{Captura del proyecto en phpStorm.}


PHPStorm es un entorno de desarrollo para PHP, cuyo propietario es IntelliJ.\cite{PHPSTORM}
He utilizado enteramente este IDE para el desarrollo del proyecto, y como IDE principal.
Como casi todos los productos de IntelliJ, no es un producto gratuito, pero jetbrains ofrece a los estudiantes y profesores licencias gratuitas.
Este IDE es de los más valorados de la comunidad y lo he escogido por las siguientes características:
\begin{itemize}
	\item Necesitaba un IDE tanto para PHP como para JavaScript, y PHPStorm me lo ofrecia. (WebStorm\cite{WEBSTORM} es un IDE especifico de JavaScript del mismo fabricante, pero phpstorm nos ofrecía lo mismo respecto a la funcionalidad).
	\item Tiene un potente control de versiones, muy intuitivo y fácil de usar respecto a otros IDEs que he trabajado (Eclipse).
	\item El IDE es muy robusto. En el desarrollo de todo el proyecto, no he visto ningún bug notable en el IDE.
\end{itemize}

https://www.jetbrains.com/phpstorm/

\subsection{Navegador Chrome}
El navegador Chrome es el navegador líder en la actualidad, con la mayor cuota de mercado. Es por eso que fue nuestra decisión depurar en él frente a otras alternativas.

El navegador Chrome nos permite realizar estas tareas como desarrolladores:
-Utilizar la consola web.
-Depurador web y javascript.
-Cambiar la resolución del dispositivo, incluso cargar resoluciones predeterminadas.
-Inspector de páginas.
-Editor CSS
-Analizador de rendimiento
-Analizador de memoria.
-Analizador de tráfico de red.
Link: https://www.google.es/chrome/browser/desktop/

\subsection{JQuery}
JQuery es una biblioteca JavaScript, con la que gracias a ella, nos permite interactuar con objetos del DOM de manera sencilla, con ciertas funcionalidades extra, la cual la convierte en la biblioteca JavaScript más usada en la actualidad. Tiene una licencia MIT por lo que podemos darle uso gratuito reconociendo al autor.

En el proyecto la utilizamos para modificación de css, efectos, animaciones y utilización de elementos del DOM.\cite{Jquery}

Link: https://jquery.com/

\subsection{Bootstrap}

Utilizamos bootstrap, una biblioteca de herramientas para frontend, por el gran contenido que nos aporta a la página, ya que abstrae al programador en ciertos elementos predefinidos en la biblioteca, entre las que destacan hojas de estilos, que a su vez están combinados con código en javascript, lo que nos aporta una gran cantidad de recursos para diseño web.
Una de las ventajas de usar bootstrap, es convertir nuestra página en una página responsive.

Link: http://getbootstrap.com/

\subsection{Chartjs}

Chartjs es la librería que hemos seleccionado para realizar nuestro proyecto. Tiene una licencia MIT, aspecto importante que vamos a explicar más adelante en nuestra memoria y anexos.\cite{ChartJS}
\imagen{chartjs2}{Banner de ChartJS. Fuente http://muchocodigo.com}

La selección de la librería para gráficos es clave en este proyecto, ya que se basa en una herramienta de graficado.
Es responsive, utiliza el canvas de HTML5, está bien documentada, es open source,tiene multitud de personalizaciones y tiene un gran soporte de la comunidad.
Link: http://www.chartjs.org/

\section{Dependencias}

\subsection{JetBrains IDE Support}

Utilizamos esta extensión de Chrome para depurar en PHPStorm mientras tenemos una página abierto en Chrome. Permite depurar en vivo la página mientras operamos con ella.

Link: https://chrome.google.com/webstore/detail/jetbrains-ide-support/hmhgeddbohgjknpmjagkdomcpobmllji?

\subsection{Bootstrap-select}
Bootstrap-select es un plugin de Jquery que utiliza bootstrap, y lo hemos empleado para todas las cajas de selección (todas las listas con texto) de nuestra herramienta de graficado.

 
\subsection{Chartjs-plugin-annotation}
Utilizamos este plugin de ChartJS, ya que permite escribir líneas y cajas dentro del canvas de la gráfica, ya que de forma nativa no te deja. Aun así, estas cajas y líneas son dibujadas a partir de datos dentro de la gráfica, por lo que hemos tenido que trabajar en una solución para que pudiera dibujar a nuestro gusto los cursores por ejemplo, desconociendo el punto exacto una vez que no es contemplada en este plugin.

Link: https://github.com/chartjs/chartjs-plugin-annotation


