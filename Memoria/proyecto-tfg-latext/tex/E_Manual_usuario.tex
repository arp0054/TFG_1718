\apendice{Documentación de usuario}

\section{Introducción}
Para este manual de usuario mostraremos el funcionamiento de la página omitiendo la instalación del servidor.
Recomendamos usar como complemento los videotutoriales que se adjuntan en el DVD del proyecto.
\section{Requisitos de usuarios}
Los requisitos para que funcione el servidor son:
\begin{itemize}
	\item PHP 5.5 \cite{PHP5.5} o PHP 5.6
	\item Servidor web, por ejemplo, apache.
\end{itemize}
Esta aplicación es compatible con los principales navegadores del mercado, entre ellos se ha verificado su uso en:
\begin{itemize}
	\item Google Chrome
	\item Mozilla Firefox
\end{itemize}

\section{Instalación}
Para emular el servidor, utilizaremos el servidor que nos genera PHPStorm, cuyos pasos ya han sido marcados en el manual del programador.
\section{Manual del usuario}

\subsection{Pantalla de inicio}
\imagen{usuario1}{Pantalla de inicio del proyecto}
En esta pantalla el usuario tendrá que seleccionar los ficheros con los que quiere trabajar, en este caso trabajaremos con 2 ficheros distintos. Para seleccionar simplemente hacer click izquierdo, y saldrá un tick verificando que hemos seleccionado dichos ficheros:
\imagen{usuario2}{Selección de archivos}
Ahora presionamos siguiente:
\imagen{usuario3}{Salir pantalla de inicio}
Ahora nuestro sistema está calculando las bibliotecas (ventana de carga), el navegador no responderá hasta que se haya calculado completamente el diccionario:
\imagen{usuario4}{Pantalla de carga}
Una vez cargadas las tramas pasaremos a tener una pantalla como esta:
\imagen{usuario5}{Pantalla Fast Graph}
Por defecto, la aplicación nos abre la opción Fast Graph.

\subsection{Fast Graph}
Vamos a dibujar nuestra primera gráfica, para ello, seleccionamos los ficheros que queremos que nos muestre de las variables y las variables que queremos graficar (ver Pantalla Fast Graph del apartado anterior).
Seleccionaremos 1 fichero y 1 variable. Aunque podemos graficar muchas más variables y seleccionar varios ficheros.

Damos al botón de save, y nos saldrá una alerta en el navegador diciendo que se han guardado los cambios.
\imagen{usuario6}{Pantalla de confirmación}

Salimos. Y se nos graficará la variable:

\imagen{usuario7}{Pantalla de confirmación}

\subsection{Añadir y modificar vistas}

Respecto al punto anterior, queremos modificar el nombre de “Left drive value FG” por otro más corto, y queremos visualizar en la misma gráfica la misma variable con un escalado de *20.

Para ello vamos a Variable Settings:
\imagen{usuario8}{Opciones de variable}

En la parte superior nos está indicando que por defecto está seleccionada esa vista de la variable.

Para ello escribimos un nombre más corto, cambiamos el color a rojo y presionamos “MODIFY”.
\imagen{usuario9}{Cambio de nombre de variable.}
Una vez dado “Modify”, sabremos que lo hemos realizado con éxito si no tenemos más vistas realizadas dando al desplegable.

Ahora desde Variable Settings añadiremos la misma variable con un escalado de * 2 y color azul. Y presionaremos el botón NEW.
\imagen{usuario10}{Creación de nueva vista de variable.}

Ahora podremos comprobar que tenemos 2 vistas creadas, pinchando en el desplegable.

Para visualizar los cambios en la gráfica pulsamos el botón exit o pinchamos fuera.

Como vemos se nos visualizarán las 2 variables:
\imagen{usuario11}{Resultado creación variables.}

En la parte superior de la gráfica, en la leyenda del gráfico, podremos quitar o poner a nuestro gusto  las variables pulsando con el botón izquierdo encima de la leyenda.

Como vemos LFG, se ha quedado muy pequeño aunque estemos sólo visualizando gracias a la función de la leyenda. Accedemos a Chart Settings y ponemos la opción de auto-axes adjustment a true.
\imagen{usuario12}{Cambio de autoajuste de ejes.}

Ahora cuando quitamos la vista “LFG*2”, se nos autoajustarán los ejes y podremos ver si problema a LFG.
\imagen{usuario13}{Resultado quitando una variable.}

\subsection{Importar y exportar vistas.}
Si queremos utilizar las mismas vistas tendremos la opción de exportar e importar dichas vistas.
Para ello seleccionamos la opción de exportar en el menú superior y nos saldrá una captura como esta:
\imagen{usuario14}{Opción de exportar vista.}

Pinchamos en el enlace y nos descargamos el fichero.

Ahora vamos a iniciar de nuevo la página pero vamos a cargar ficheros diferentes. Cuando nos salga la opción Fast Graph damos exit, y seleccionamos en el menu superior la opción “Import”. 
Seleccionamos los ficheros en los que queremos aplicar la vista que vamos a importar y seleccionamos el fichero que deseamos:
\imagen{usuario15}{Opción de importar vista.}

Damos al botón de import, y nos saldrá una alerta diciendo que se han guardado los cambios.

Pulsamos exit y observamos los cambios:
\imagen{usuario16}{Solución al importar una vista.}

\subsection{Creación de gráficas sin Fast Graph.}
De nuevo iniciamos la página y seleccionamos un fichero.
Descartamos la opción de Fast Graph.
Y presionamos en chart settings.
Esta vez vamos a representar una gráfica XY, para ello seleccionamos los ficheros que queremos trabajar y seleccionamos en el Chart Type “ X Y Chart”.
Damos el botón guardar (nos saldrá una alerta de que se han guardado cambios).
\imagen{usuario18}{Opciones gráficos.}

Ahora vamos a variable settings e introducimos los 2 valores que queremos visualizar en el eje x y, con su respectivo nombre y sus respectivas opciones. Damos botón guardar (nos saldrá una alerta). Después daremos el botón exit o pulsaremos fuera.
\imagen{usuario19}{Creación variable XY.}

Se nos graficará la gráfica XY.
\imagen{usuario20}{Solución gráfica XY.}

\subsection{Configuración del XML.}

Tenemos la opción de aplicar un XML distinto siempre que nosotros queramos al de defecto. Para ello damos en el menú la opción de XML configuration y podemos cargar otro xml con distinta configuración de tramas CAN. Nosotros utilizaremos el xml que hay en nuestro proyecto llamado left drive value.
\imagen{usuario21}{Cambiar XML.}

Hay que tener en cuenta que puede tardar un tiempo pues tiene que re-calcular toda la biblioteca.
\imagen{usuario22}{Único resultado en variables.}

Ahora en fast graph, en este ejemplo, la opción de graficar Left drive value será la única opción disponible dentro de las variables.
\imagen{usuario23}{Resultado con único resultado.}

\subsection{Cursores.}
Para utilizar los cursores, simplemente tenemos que pinchar en la parte superior izquierda escogiendo que cursor queremos utilizar (verde o azul). 
Una vez seleccionado el cursor, se nos quedará como fijado y podremos dibujarlo en la gráfica, presionando el click izquierdo las veces que nosotros consideremos oportuno hasta ajustar el punto exacto en el que queremos dicho punto.

Para seleccionar el segundo cursor, pinchamos de nuevo en el cursor seleccionado y pinchamos en el otro. Si dibujamos el segundo cursor (click izquierdo gráfica), nos creará automáticamente una tabla con la diferencia y suma de los 2 cursores.

\imagen{usuario24}{Resultado con único resultado.}

\subsection{Utilizar la ventana temporal.}
En la ventana temporal disponemos de un zoom horizontal. Para utilizar dicho zoom haremos lo siguiente:
Limite izquierdo y derecho del zoom:

\imagen{usuario25}{Ventana temporal: en rojo zoom izquierdo, azul derecho.}


Como vemos en la imagen en el cuadrado rojo tendremos el zoom izquierdo, y en el cuadrado derecho el zoom derecho.
Para cambiar el zoom, pincharemos en uno de los 2 y podremos modificar el limite de ese lado pinchando en la gráfica inferior o utilizando los botones de desplazamiento. Hay que tener en cuenta que cuando utilizamos el click en la gráfica, se aproximará al punto más cercano.
Así podremos modificar a nuestro gusto la parte que queremos visualizar de la gráfica.
\imagen{usuario26}{Zoom con ventana temporal.}

Recordamos que cada vez que se hace zoom se hace un diezmado automático, lo que nos permite ver puntos que antes no podiamos ver sin realizar el zoom.

Dentro de las otras opciones de la ventana temporal tenemos un modo de reproducción, donde la gráfica se irá reproduciendo y podremos visualizar el punto actual gracias a ese zoom inferior.
Además podremos incrementar la velocidad de reproducción pinchando en 1X, de mayor a menor entre las distintas opciones.
Con el botón pause podremos pausar la reproducción de la gráfica.
\imagen{usuario27}{Modo reproducción.}

Con el botón STOP volveremos al punto por defecto del zoom.


\subsection{Utilizar búsquedas y filtros}
Esta herramienta de graficado permite realizar búsquedas y filtros aplicando una consulta (ver en la memoria la sintaxis de las consultas).
Con el ejemplo anterior realizaremos un filtrado y una búsqueda.

Si presionamos el primer icono de consulta (filtro) se filtrarán dichos datos y solo veremos los que se cumplen. El resultado en este ejemplo es:
\imagen{usuario28}{Filtro en la parte inferior.}
Si presionamos borrar filtro o de nuevo al botón de filtro, se eliminará el filtro.

La opción más interesante se presenta con el botón buscar.
Con la misma consulta, nos situaría en el primer punto que se cumpliría con un cursor rojo.
A continuación nosotros tenemos la opción de hacer nuevas consultas partiendo de ese cursor y seleccionando la flecha de izquierda o derecha para calcular el primer punto respecto a ese cursor.
\imagen{usuario29}{Búsqueda con zoom automático.}

Si no se trata del primer punto, la herramienta hará un zoom automático para poder ver mejor ese punto.
Por ello, con el segundo punto realizaría un zoom.

También tenemos la opción de asignar los cursores anteriores al punto de la busqueda.

Por ejemplo, asignamos el cursor verde a la primera consulta y el azul a la segunda(consulta ldv = 91 y la otra ldv=94).Y obteniendo el segundo punto, dando el botón stop para tener una visión global:
\imagen{usuario30}{Asignación cursores con el cursor de búsqueda.}



