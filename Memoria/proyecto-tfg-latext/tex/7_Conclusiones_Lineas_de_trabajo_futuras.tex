\capitulo{7}{Conclusiones y Líneas de trabajo futuras}

Para concluir con la memoria de este proyecto, presentaré en primer lugar las posibles líneas de trabajo futuras y finalizaré con una conclusión del mismo.

\section{Líneas de trabajo futuras}
A continuación, propongo un par de lineas futuras para continuar con esta aplicación, que no fueron posible realizar por cuestión de tiempo:
\begin{itemize}
	\item Que la aplicación permita leer datos de otras fuentes, como por ejemplo de una base de datos, para poder realizar y monitorear gráficas en tiempo real.
	\item Que las ventanas sean escalables, y personalizables, de tal modo, que el usuario pueda hacer pequeñas modificaciones o eliminar ciertos elementos que se presenten en la gráfica.
	\item Crear una aplicación universal en la red, que te permita guardar el estado actual de una gráfica y poder compartir el enlace. Así como seleccionar archivos y distintas disposiciones de las tramas.
	\item Migrar la aplicación web a escritorio.
\end{itemize}

\section{Conclusiones}

En este proyecto, a nivel personal, me he enfrentado a muchos retos. 
\begin{itemize}
	\item Por un lado, me he dado cuenta, que independiente de las tecnologías que conozcas, tienes que adaptarte en cada proyecto a la situación real. Por ejemplo, no puede implantar un sistema de base de datos, porque el dispositivo no tenía la capacidad de procesamiento necesaria para ello. Y tuve que realizar todo el proyecto en estructuras de datos.
 	\item También me he dado cuenta, de la importancia que han tenido para mi ciertas asignaturas de la carrera, que en principio pensaba que no tenían mucha utilidad práctica. Por ejemplo, he utilizado en un grado elevado los conocimientos adquiridos en la asignatura de procesadores del lenguaje, a la hora de analizar (parsear) el fichero XML o en crear una sintaxis para la búsqueda y los filtros.
 	
En general pienso que este proyecto tenía un trabajo y dificultad elevado, pues ciertos requisitos eran técnicamente complicados y puntillosos, como pudo ser el filtrado de datos, o la reproducción de la gráfica, por lo que tuve que documentarme e investigar en muchos de ellos varios días sin poder obtener ningún resultado.

Personalmente pienso que he realizado un buen trabajo con la herramienta de graficado, y que he cumplido satisfactoriamente en un tiempo razonable todos los requisitos propuestos por mi tutor, que en muchos casos se fueron ampliando en el tiempo.

También agradezco poder ver de primera mano cómo confluyen el mundo de la informática con el mundo industrial, y qué he sido capaz de realizar con mis conocimientos una vez visto el resultado final.

\end{itemize}

Por último, valoro positivamente la experiencia en la realización de este proyecto; el conocimiento plasmado en el proyecto y espero poder seguir desarrollando proyectos de diversa índole para poder progresar como desarrollador de software.


