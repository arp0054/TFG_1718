\capitulo{1}{Introducción}

En esta memoria se muestra el desarrollo de una herramienta gráfica que sirve para monitorizar variables, como ya hemos mencionado en la introducción, estará alojada en un servidor web local, (lightppd) dentro del mismo dispositivo. 


Dicho servidor es un equivalente a un servidor apache con php y donde la aplicación estará enteramente programada en JavaScript y HTML5, todo esto dentro del datalogger/caja negra.


Únicamente usaremos php para mapear los ficheros disponibles en la caja negra que queramos graficar.


Una vez que tenemos los archivos mapeados, la caja negra que tiene otros servicios corriendo(recordamos que es un dispositivo con linux embebido muy limitado) se encargará únicamente de enviar los archivos necesarios al cliente, entre ellos el contenido de la página y archivos javascript. De esa forma nos aseguramos que la carga de CPU del datalogger es mínima. 


Esto es imprescindible, porque al ser un elemento industrial, no nos podemos permitir que se detenga o se congele la máquina por culpa de sobrecarga en la CPU o en memoria, ya que se pararía toda la producción y funcionamiento de todos los elementos que sean controlados o gestionados por ella.

\imagen{datalogger}{Imagen del dispositivo en cuestión. Fuente fabricante: http://www.matrix.es}

Es por ello, que la herramienta para realizar todos los cálculos, extracción de datos y funcionalidades está programada en JavaScript, el cual, deja todo el peso de ejecución a la maquina cliente.

\imagen{funcJavaScript}{Diagrama funcionamiento de un proyecto con JavaScript. Fuente: https://openclassrooms.com}

La máquina cliente es la encargada de cargar la biblioteca de variables. En primer lugar, tenemos que parsear el XML maestro, el cual, contiene la disposición en bits de la trama. Éste está codificado en hexadecimal en formato littleEndian.


Una vez extraídas la disposición de las variables gracias al XML, analizaremos todos los ficheros seleccionados y crearemos una biblioteca con esas variables.

Una vez obtenidas dichas variables, junto a todos los datos obtenidos correspondientes a los ficheros, el usuario podrá invocarlas a una gráfica con distintas vistas y opciones.

Esta gráfica ha sido optimizada para que sólo se puedan cargar un máximo de puntos, de manera que vayan de forma fluida en el navegador, por lo tanto, se realiza un diezmado de datos. No se desprecia ningún punto, puesto que la gráfica tiene la funcionalidad, gracias a la ventana temporal, de hacer zoom y calcular de nuevo un diezmado, mostrando hasta ese cierto máximo de puntos fijado.

Aparte de esas funcionalidades, la herramienta tiene otras como puede ser un modo de reproducción, filtrados, búsquedas, cursores, comparación en tabla de varios cursores, modo XY, umbrales, escalados, entre otras, a disposición del usuario; de las que hablaremos de ellas en esta memoria con más profundidad.