\documentclass[a4paper,12pt,twoside]{memoir}
% Castellano
\usepackage[spanish,es-tabla]{babel}
\selectlanguage{spanish}
\usepackage[utf8]{inputenc}
\usepackage[T1]{fontenc}
\usepackage{lmodern} % Scalable font
\usepackage{microtype}
\usepackage{placeins}
\usepackage{hyperref}

\RequirePackage{booktabs}
\RequirePackage[table]{xcolor}
\RequirePackage{xtab}
\RequirePackage{multirow}



% Ecuaciones
\usepackage{amsmath}

% Rutas de fichero / paquete
\newcommand{\ruta}[1]{{\sffamily #1}}

% Párrafos
\nonzeroparskip


% Imagenes
\usepackage{graphicx}
\newcommand{\imagen}[2]{
	\begin{figure}[!h]
		\centering
		\includegraphics[width=0.9\textwidth]{#1}
		\caption{#2}\label{fig:#1}
	\end{figure}
	\FloatBarrier
}

\newcommand{\imagenflotante}[2]{
	\begin{figure}%[!h]
		\centering
		\includegraphics[width=0.9\textwidth]{#1}
		\caption{#2}\label{fig:#1}
	\end{figure}
}



% El comando \figura nos permite insertar figuras comodamente, y utilizando
% siempre el mismo formato. Los parametros son:
% 1 -> Porcentaje del ancho de página que ocupará la figura (de 0 a 1)
% 2 --> Fichero de la imagen
% 3 --> Texto a pie de imagen
% 4 --> Etiqueta (label) para referencias
% 5 --> Opciones que queramos pasarle al \includegraphics
% 6 --> Opciones de posicionamiento a pasarle a \begin{figure}
\newcommand{\figuraConPosicion}[6]{%
  \setlength{\anchoFloat}{#1\textwidth}%
  \addtolength{\anchoFloat}{-4\fboxsep}%
  \setlength{\anchoFigura}{\anchoFloat}%
  \begin{figure}[#6]
    \begin{center}%
      \Ovalbox{%
        \begin{minipage}{\anchoFloat}%
          \begin{center}%
            \includegraphics[width=\anchoFigura,#5]{#2}%
            \caption{#3}%
            \label{#4}%
          \end{center}%
        \end{minipage}
      }%
    \end{center}%
  \end{figure}%
}

%
% Comando para incluir imágenes en formato apaisado (sin marco).
\newcommand{\figuraApaisadaSinMarco}[5]{%
  \begin{figure}%
    \begin{center}%
    \includegraphics[angle=90,height=#1\textheight,#5]{#2}%
    \caption{#3}%
    \label{#4}%
    \end{center}%
  \end{figure}%
}
% Para las tablas
\newcommand{\otoprule}{\midrule [\heavyrulewidth]}
%
% Nuevo comando para tablas pequeñas (menos de una página).
\newcommand{\tablaSmall}[5]{%
 \begin{table}
  \begin{center}
   \rowcolors {2}{gray!35}{}
   \begin{tabular}{#2}
    \toprule
    #4
    \otoprule
    #5
    \bottomrule
   \end{tabular}
   \caption{#1}
   \label{tabla:#3}
  \end{center}
 \end{table}
}

%
% Nuevo comando para tablas pequeñas (menos de una página).
\newcommand{\tablaSmallSinColores}[5]{%
 \begin{table}[H]
  \begin{center}
   \begin{tabular}{#2}
    \toprule
    #4
    \otoprule
    #5
    \bottomrule
   \end{tabular}
   \caption{#1}
   \label{tabla:#3}
  \end{center}
 \end{table}
}

\newcommand{\tablaApaisadaSmall}[5]{%
\begin{landscape}
  \begin{table}
   \begin{center}
    \rowcolors {2}{gray!35}{}
    \begin{tabular}{#2}
     \toprule
     #4
     \otoprule
     #5
     \bottomrule
    \end{tabular}
    \caption{#1}
    \label{tabla:#3}
   \end{center}
  \end{table}
\end{landscape}
}

%
% Nuevo comando para tablas grandes con cabecera y filas alternas coloreadas en gris.
\newcommand{\tabla}[6]{%
  \begin{center}
    \tablefirsthead{
      \toprule
      #5
      \otoprule
    }
    \tablehead{
      \multicolumn{#3}{l}{\small\sl continúa desde la página anterior}\\
      \toprule
      #5
      \otoprule
    }
    \tabletail{
      \hline
      \multicolumn{#3}{r}{\small\sl continúa en la página siguiente}\\
    }
    \tablelasttail{
      \hline
    }
    \bottomcaption{#1}
    \rowcolors {2}{gray!35}{}
    \begin{xtabular}{#2}
      #6
      \bottomrule
    \end{xtabular}
    \label{tabla:#4}
  \end{center}
}

%
% Nuevo comando para tablas grandes con cabecera.
\newcommand{\tablaSinColores}[6]{%
  \begin{center}
    \tablefirsthead{
      \toprule
      #5
      \otoprule
    }
    \tablehead{
      \multicolumn{#3}{l}{\small\sl continúa desde la página anterior}\\
      \toprule
      #5
      \otoprule
    }
    \tabletail{
      \hline
      \multicolumn{#3}{r}{\small\sl continúa en la página siguiente}\\
    }
    \tablelasttail{
      \hline
    }
    \bottomcaption{#1}
    \begin{xtabular}{#2}
      #6
      \bottomrule
    \end{xtabular}
    \label{tabla:#4}
  \end{center}
}

%
% Nuevo comando para tablas grandes sin cabecera.
\newcommand{\tablaSinCabecera}[5]{%
  \begin{center}
    \tablefirsthead{
      \toprule
    }
    \tablehead{
      \multicolumn{#3}{l}{\small\sl continúa desde la página anterior}\\
      \hline
    }
    \tabletail{
      \hline
      \multicolumn{#3}{r}{\small\sl continúa en la página siguiente}\\
    }
    \tablelasttail{
      \hline
    }
    \bottomcaption{#1}
  \begin{xtabular}{#2}
    #5
   \bottomrule
  \end{xtabular}
  \label{tabla:#4}
  \end{center}
}



\definecolor{cgoLight}{HTML}{EEEEEE}
\definecolor{cgoExtralight}{HTML}{FFFFFF}

%
% Nuevo comando para tablas grandes sin cabecera.
\newcommand{\tablaSinCabeceraConBandas}[5]{%
  \begin{center}
    \tablefirsthead{
      \toprule
    }
    \tablehead{
      \multicolumn{#3}{l}{\small\sl continúa desde la página anterior}\\
      \hline
    }
    \tabletail{
      \hline
      \multicolumn{#3}{r}{\small\sl continúa en la página siguiente}\\
    }
    \tablelasttail{
      \hline
    }
    \bottomcaption{#1}
    \rowcolors[]{1}{cgoExtralight}{cgoLight}

  \begin{xtabular}{#2}
    #5
   \bottomrule
  \end{xtabular}
  \label{tabla:#4}
  \end{center}
}


\graphicspath{ {./img/} }

% Capítulos
\chapterstyle{bianchi}
\newcommand{\capitulo}[2]{
	\setcounter{chapter}{#1}
	\setcounter{section}{0}
	\chapter*{#2}
	\addcontentsline{toc}{chapter}{#2}
	\markboth{#2}{#2}
}

% Apéndices
\renewcommand{\appendixname}{Apéndice}
\renewcommand*\cftappendixname{\appendixname}

\newcommand{\apendice}[1]{
	%\renewcommand{\thechapter}{A}
	\chapter{#1}
}

\renewcommand*\cftappendixname{\appendixname\ }

% Formato de portada
\makeatletter
\usepackage{xcolor}
\newcommand{\tutor}[1]{\def\@tutor{#1}}
\newcommand{\course}[1]{\def\@course{#1}}
\definecolor{cpardoBox}{HTML}{E6E6FF}
\def\maketitle{
  \null
  \thispagestyle{empty}
  % Cabecera ----------------
\noindent\includegraphics[width=\textwidth]{cabecera}\vspace{1cm}%
  \vfill
  % Título proyecto y escudo informática ----------------
  \colorbox{cpardoBox}{%
    \begin{minipage}{.8\textwidth}
      \vspace{.5cm}\Large
      \begin{center}
      \textbf{TFG del Grado en Ingeniería Informática}\vspace{.6cm}\\
      \textbf{\LARGE\@title{}}
      \end{center}
      \vspace{.2cm}
    \end{minipage}

  }%
  \hfill\begin{minipage}{.20\textwidth}
    \includegraphics[width=\textwidth]{escudoInfor}
  \end{minipage}
  \vfill
  % Datos de alumno, curso y tutores ------------------
  \begin{center}%
  {%
    \noindent\LARGE
    Presentado por \@author{}\\ 
    en Universidad de Burgos --- \@date{}\\
    Tutor: \@tutor{}\\
  }%
  \end{center}%
  \null
  \cleardoublepage
  }
\makeatother

\newcommand{\nombre}{Álvaro Ruifernández Palacios} %%% cambio de comando

% Datos de portada
\title{Detección de objetos con láser de Seguridad}
\author{\nombre}
\tutor{Jesús Enrique García Sierra}
\date{\today}

\begin{document}

\maketitle


\newpage\null\thispagestyle{empty}\newpage


%%%%%%%%%%%%%%%%%%%%%%%%%%%%%%%%%%%%%%%%%%%%%%%%%%%%%%%%%%%%%%%%%%%%%%%%%%%%%%%%%%%%%%%%%%%%%%%%%%%%%%%%%%%%%%%%%%%%%%%%%%%%%%%%%%%%%%%%%%%%%%%%%%%%%%%%%%%%%%
\thispagestyle{empty}


\noindent\includegraphics[width=\textwidth]{cabecera}\vspace{1cm}

\noindent D. Jesús Enrique García Sierra, profesor del departamento de Ingeniería Civil, área de Sistemas de la información.

\noindent Expone:

\noindent Que el alumno D. \nombre, con DNI 71365383V, ha realizado el Trabajo final de Grado en Ingeniería Informática titulado Detección de objetos con láser de Seguridad. 

\noindent Y que dicho trabajo ha sido realizado por el alumno bajo la dirección del que suscribe, en virtud de lo cual se autoriza su presentación y defensa.

\begin{center} %\large
En Burgos, {\large \today}
\end{center}

\vfill\vfill\vfill

% Author and supervisor
\begin{minipage}{0.45\textwidth}
\begin{flushleft} %\large
Vº. Bº. del Tutor:\\[2cm]
D. nombre tutor
\end{flushleft}
\end{minipage}
\hfill
\begin{minipage}{0.45\textwidth}
\begin{flushleft} %\large
Vº. Bº. del co-tutor:\\[2cm]
D. nombre co-tutor
\end{flushleft}
\end{minipage}
\hfill

\vfill

% para casos con solo un tutor comentar lo anterior
% y descomentar lo siguiente
%Vº. Bº. del Tutor:\\[2cm]
%D. nombre tutor


\newpage\null\thispagestyle{empty}\newpage


\frontmatter

% Abstract en castellano
\renewcommand*\abstractname{Resumen}
\begin{abstract}
Los vehículos autoguiados o AGVs son una de las grandes innovaciones en el sector de la industria de estos últimos años. Estos son capaces de cumplir trabajos de traslado y colocación de mercancías en los distintos sectores de la industria en la que se encuentran instalados sin la necesidad de operarios que los manejen.\\
\\
Para realizar estas tareas, además de los sistemas de guiado, necesitan un sistema de detección de objetos en su entorno, tanto para localizar los elementos a transportar como para detectar obstáculos . Para esta tarea los estos sistemas llevan incorporado un láser de seguridad el cual hace lecturas periódicas de su entorno.\\
\\
Dos de los problemas que poseen estos láseres son: el número de campos de área posibles y la exactitud de las detecciones. Estos láseres tienen como característica la posibilidad un número limitado de campos de área en los cuales solo se detectan las variaciones en cuanto a los puntos detectados por el láser pero sin especificar donde se encuentra el objeto exactamente, es decir, devuelven la detección de un objeto en un campo pero no en que parte del mismo se encuentra.\\
\\
En este proyecto se ha desarrollado un software destinado al láser para detectar objetos de manera mucho más precisa de lo que lo hacía hasta ahora los láseres y poder detectar estos objetos en un número de campos de área casi ilimitado, eliminando también la primera de las restricciones descritas anteriormente.
\end{abstract}

\renewcommand*\abstractname{Descriptores}
\begin{abstract}
Láser, AGV, vehículos autoguiados,detección de objetos, Python,Spyder,Jupyter Notebook\ldots
\end{abstract}

\clearpage

% Abstract en inglés
\renewcommand*\abstractname{Abstract}
\begin{abstract}
Automated guided vehicle o AGVs are one of the most important innovations in the industry of the last few years. These are able to complete labors of transfer and placement of goods in the different sectors of the industry they are installed without any driver.\\
\\
To perform these tasks, apart from the guidance systems, they need a object detection system in their environment, both to locate the goods to be transported and to detect obstacles. Thus, these systems incorporate a safety laser which makes periodic measurements of their surroundings. \\
\\
Some of the problems or aspects to improve these lasers have are 2: the limited possible areas and the accuracy of measurements.One of the problems these lasers have is the limited number of areas which only detect variations in the measurements detected by the laser with no specification about where exactly the object is.\\
\\
In this project software was developed for a laser to detect objects far more accurate than current lasers and be able to detect these objects in a almost unlimited number of areas, also deleting the first of the restrictions previously described.
\end{abstract}

\renewcommand*\abstractname{Keywords}
\begin{abstract}
Láser, AGV,automated guided vehicle,object detection, Python,Spyder,Jupyter Notebook\ldots
\end{abstract}

\clearpage

% Indices
\tableofcontents

\clearpage

\listoffigures

\clearpage

\listoftables
\clearpage

\mainmatter
\capitulo{1}{Introducción}

Los AGV \cite{AGVdes} son una de las últimas innovaciones en el mundo de la industria. Son robots autónomos que sustituyen al trabajo humano en el sentido de transporte de objetos o mercancías entre las distintas secciones de las industrias. \\
\\
Para la realización de sus propósitos estos autómatas, además de sus sistemas de guiado, necesitan estructuras que les permitan reconocer si en su camino existe algún obstáculo o si el objeto con el que trabaja (mercancía) se encuentra cerca de ellos. Para esto, los AGVs llevan incorporado, como norma general, un láser de seguridad, el cual recoge un gran conjunto de datos sobre el entorno que lo rodea.\\
\\
Actualmente, esta tecnología se basa en la configuración de áreas en las cuales el láser detecta las variaciones de puntos que se localizan dentro de estas, representado estas variaciones de forma booleana a través de una alarma. Los láseres de seguridad poseen un número limitado de estas áreas, también denominadas campos de seguridad, las cuales, como bien se ha dicho antes, solo ofrecen información a cerca de si un objeto es detectado o no en un determinado campo.\\
\\
En este proyecto se desarrollará un sistema con el que se pretende solventar los problemas anteriormente descritos, ya que permitirá un número de áreas prácticamente ilimitado y la posibilidad de conocer de forma exacta la posición de los objetos que el láser detecte en su entorno, lo cual no se ha hecho hasta ahora con este tipo de láseres.\\
\\

\capitulo{2}{Objetivos del proyecto}

En este apartado se explican los distintos objetivos identificados en este proyecto, distinguiendo entre los objetivos generales del proyecto y los objetivos técnicos.

\section{Objetivos generales}

Se tiene por objetivo crear un software capaz de realizar un análisis de los datos recibidos del láser para identificar los objetos de su entorno. Por lo tanto, los objetivos generales de este proyecto son:
\begin{itemize}
	\item Crear una conexión para la comunicación con el láser
	\item Intercambiar mensajes con el láser para obtener sus datos de lectura.
    \item Realizar un análisis de dichos datos para comprobar si existe o no un objeto en las áreas requeridas por el usuario.
\end{itemize} 

\section{Objetivos funcionales}
Los objetivos funcionales estimados para este proyecto son:
\begin{itemize}
	\item El usuario podrá escoger el número de zonas que desea analizar, siendo este número tan grande como desee.
	\item El usuario decidirá cada una de las coordenadas en las que se encuentran los límites de las áreas que pretende analizar.
	\item El usuario podrá ejecutar cuantas veces quiera el programa para poder observar diferentes lecturas.
	\item El usuario podrá observar una serie de mensajes "true" o "false" (en el orden de inserción de las áreas anteriormente establecidas) para comprobar en qué áreas se ha detectado un objeto y en cuales no.
	\item El usuario podrá observar de forma gráfica (más concretamente en un gráfico de puntos) los resultados del análisis y la posición de las áreas establecidas.
\end{itemize}

\section{Objetivos técnicos}

Los objetivos técnicos planteados para este proyecto son:
\begin{itemize}
	\item Aprender el uso de Python para la creación del código del proyecto, a través de todas sus funcionalidades previstas con el objetivo de la lectura de datos de las lecturas del láser.
	\item Usar librerías de Python no empleadas hasta el momento en ninguna asignatura del grado como binascii o re, las cuales han servido para poder manejar los datos recibidos y desarrollar el comportamiento deseado.
	\item Encontrar todos los elementos hardware necesarios para la creación del sistema hardware necesario para el proyecto.
	\item Crear un proyecto el cual implementa la funcionalidad que se requiere en los objetivos del proyecto, es decir, que realice la lectura, tratamiento y análisis de los datos del láser.
	\item Utilizar un sistema de control de versiones como es Git, más concretamente uno de sus servicios centrales más utilizados conocido como GitHub. Esta herramienta es útil en el caso de producirse algún error grave en alguna de las partes del proyecto ya que se puede restaurar la versión previa (la cual se considera estable).
\end{itemize}


\capitulo{3}{Conceptos teóricos}

En este proyecto se deben de emplear una gran cantidad de  conocimientos adquiridos a lo largo de las asignaturas del grado. En esta sección se describirán cuales son esos conceptos además de la forma en la que se aplican en este proyecto.

\section{Organización de la infraestructura}

Para la implementación del sistema que nos va a permitir el desarrollo del sistema que tiene como objetivo este proyecto se necesitarán tres elementos principales:
\begin{itemize}
	 \item Láser: esta parte es la que se encarga de obtener los datos de su entorno. Estos datos serán enviados a través de una interfaz hacia la unidad de tratamiento de datos. En este caso, el láser del que se dispone, y por lo tanto con el que se va a desarrollar el proyecto es el Hokuyo Safety Laser Scanner (UAM-05LP-T301). Este aparato es capaz de desarrollar tres áreas de detección dependiendo de la distancia a la que detecte un elemento (puede ser de 20, 10 y 5 metros). Aunque sea este láser con el que se va a realizar el proyecto, el principal objetivo es el de poder hacer que el programa pueda funcionar en diferentes tipos de láser con diferentes formas de obtención de datos.\\
    \item Cable ethernet: es el soporte a través del cual se transportan tanto los mensajes enviado por el ordenador hacia el láser como las respuestas y los datos de lectura que este manda al PC demandante. Esta es la que, en la descripción del anterior elemento se conoce como la interfaz. En el caso de este proyecto se usará un UAM-NET, un cable Ethrernet de 3 metros de longitud desarrollado por Hokuyo, misma empresa desarrolladora del láser empleado, lo que hace que resulte idóneo para evitar problemas de incompatibilidad y asegurar así el correcto funcionamiento del sistema.
    \item PC: es la parte principal del proyecto, ya que es la encargada de permitir al usuario tanto introducir las áreas donde se necesita detectar los objetos, como las órdenes que el usuario desea transmitir al láser para recibir la información. Además también es la encargada de recibir y procesar las respuestas del láser para poder mostrar al usuario los resultados de la comparación de los datos y las coordenadas de las áreas introducidas. La utilización de este aparato es debida a que este proyecto sería destinado a ser introducido dentro de un AGV pero se necesita hacer visible al usuario a través de una pantalla.\\ 
\end{itemize}

\subsection{Implementación final en el AGV}
Con esta estructura, el sistema puede ser implementado en AGVs que no posean movimiento, es decir, aquellos cuyas partes móviles no hagan desplazar toda la estructura del vehículo, ya que se necesita un ordenador para visualizar los resultados del análisis.

\imagen{diagrmaExt3}{Implementación externa}

Esta primera opción de implementación tiene ciertas ventajas pero también posee ciertos inconvenientes:
Ventajas:
\begin{enumerate}
	\item Visualización de la gráfica que representa tanto las lecturas del láser como el resultado del análisis de forma más visual y fácil de analizar por el usuario.
	\item Facilidad para el cambo de coordenadas de cada uno de los límites que conforman cada una de las áreas a analizar.
\end{enumerate}
Desventajas:
\begin{enumerate}
	\item Necesidad de un PC externo que realice el análisis de datos.
\end{enumerate}

Aunque si se obvia esta característica de la detección de objetos (reduciendo ese proceso a datos booleanos) si que es posible emplear el sistema con AVGs móviles.
\imagen{diagrmaInt3}{Implementación interna}

Aunque esta opción tambien tiene sus ventajas e inconvenientes, al igual que la anterior:
Ventajas:
\begin{enumerate}
	\item Independencia de ningún PC externo que realice el análisis.
	\item Reducción de utilización de recursos (tanto hardware como software) en la ejecución del sistema, debido a la omisión de la representación gráfica.
\end{enumerate}
Desventajas:
\begin{enumerate}
	\item Incapacidad de visualización de los resultados del análisis debido a la ausencia de representación antes mencionada.
\end{enumerate}
\section{Sistemas distribuidos}

Se denomina sistema distribuido a aquel sistema en el que cada equipo se conecta con el resto de los integrantes del sistema para poder compartir sus recursos además de poder utilizar los recursos que comparten el resto de los equipos conectados para realizar procesos para los cuales se necesitan una cantidad de recursos que mayor de la que el usuario de cada equipo posee de forma local. Otra forma de definir este concepto es la ejecución de procesos desde un equipo en otro cuyas características se adapten mejor a las del objetivo que se desea alcanzar.\\
En el caso de este proyecto, la forma en la que se emplea el concepto de sistema distribuido es la que se adapta a la última de las definiciones antes mencionadas. Esto se debe a que el PC se comunica vía cable ethernet con el láser ya que es el único elemento que puede realizar los procesos de lectura de área y devolver al PC los datos correspondientes o, en defecto de estos, el mensaje de error correspondiente.\\
En este caso se establece una estructura de cliente- servidor, siendo cada uno de ellos el PC y el láser respectivamente. Esto se demuestra al observar que el funcionamiento se basa en el envío de comandos desde el PC al láser (petición) y este responde con los datos de lectura o de error (respuesta).

\section{Procesado de datos}

La minería de datos es la ciencia a través de la cual se analiza un gran conjunto de datos para poder descubrir características, patrones o comportamientos ocultos a primera vista.\\
Cuando el programa desarrollado se comunica con el láser, este debe recibir las tramas que el láser crea al analizar su área de observación, las cuales deben de ser tratadas para separar la información importante de las cabeceras y demás datos que el láser usa para interactuar con su aplicación. Una vez hemos conseguido extraer ese tipo de datos, hemos de descartar aquellos datos correspondientes a las zonas donde el láser no detecta ningún objeto, quedándonos solo con aquellas lecturas con valor significativo para la función principal del programa. Tras esto, se analizarán los datos para ver si se encuentran dentro de los límites establecidos por cada una de las áreas a analizar según las preferencias del usuario.\\

\subsection{Comandos disponibles}

Para que el láser devuelva los datos de sus lecturas se le debe de mandar un comando específico dependiendo de la forma de lectura que deseamos que realice. Hay muchos tipos de comandos disponibles para enviar al láser pero para los objetivos de este proyecto vamos a emplear un único tipo, los comandos AR.\\
Estos comandos son los que le indican al láser la orden de enviar los datos de lectura. Dependiendo del comando AR enviado, se enviará un determinado conjunto de datos u otro. Algunos de estos comandos sirven para detener el funcionamiento provocado por otros comandos del mismo tipo. Los comandos de este tipo son 6:\\
\begin{enumerate}
	\item AR00: envía la medición de las distintas distancias de una única lectura del entorno.
	\item AR01: envía las mediciones de las distintas distancias e intensidades de una única lectura del entorno.
	\item AR02: envía las mediciones de las distintas distancias de cada una de las lecturas del entorno que realiza de forma contínua.
	\item AR03: sirve para detener el comportamiento continuado del comando AR02, se recibe después un mensaje de confirmación.
	\item AR04: envía las mediciones de las distintas distancias e intensidades de cada una de las lecturas del entorno que realiza de forma contínua.
	\item AR05: sirve para detener el comportamiento continuado del comando AR04, se recibe después un mensaje de confirmación.
\end{enumerate}
\bigskip
\tablaSmall{Resumen de las características de cada comando AR}{l c c c c c}{coamndoscaracteristicas}
{ \multicolumn{1}{l}{Comandos} & Continuo & Unico & Distancia & Distancia-Intensidad  & Parada \\}{
AR00 & & X & X & &\\
AR01 & & X & & X &\\
AR02 & X & & X & &\\
AR03 & & & & & X\\
AR04 & X & & & X &\\
AR05 & & & & & X\\
}
 \subsection{Envío y recepción de mensajes}
 Tras haber escogido el comando a utilizar, se debe construir el mensaje que se va a enviar desde el PC hacia el láser. Para esta construcción se debe de seguir una estructura específica basada en estos campos.\\
\imagen{estructuraMsgEnv}{Mensaje de petición al láser}
 \begin{itemize}
 	\item STX: Carácter, habitualmente correspondiente a "2" codificado como Bytes, que indica al láser el comienzo del mensaje que se le desea enviar.
 	\item Command size: como su propio nombre indica, es el tamaño del mensaje que el láser va a recibir incluyendo los caracteres de inicio y fin. Para los comandos que se van a emplear en este proyecto, los 4 caracteres que representan esta parte del mensaje van a ser siempre "000E" ya que siempre tienen la misma extensión de 14 caracteres.
 	\item Header: o cabecera, es la parte del mensaje que indica el tipo de comando que se está mandando desde el ordenador. En este caso "AR".
 	\item Subheader: después de haber expresado el tipo de comando, se especifica en te campo que comando del tipo escogido se va a enviar.
 	\item CRC (Cyclical Redundancy Checking):  o verificación de redundancia cíclica, es un código que se añade al mensaje para que e láser verifique la ausencia de fallos en la creación o transmisión del mensaje recibido. Existen muchos tipos de CRC, pero en el caso del láser empleado se usa el CRC Kermit.
 	\item EXT: Carácter, habitualmente correspondiente a "3" codificado como Bytes, que indica al láser el final del mensaje que se le desea enviar.
 \end{itemize}
 
Tras recibir este mensaje, el láser analiza este mensaje, para comprobar que no falte ninguna de las partes descritas anteriormente y la corrección del CRC (evitando fallos de trasmisión). El mensaje de respuesta también obedece a una estructura muy estricta:\\
\imagen{estructuraMsgRcv}{Mensaje de respuesta del láser}
\begin{itemize}
	\item STX y ETX: Son los caracteres de inicio y final del mensaje enviado pro el láser. su definición es identica a la de sus homólogos en el anterior mensaje descrito.
	\item Reply size: al igual que en el anterior se especifica el tamaño del mensaje en 4 caracteres que representan este valor en hexadecimal aunque a diferencia con esta, este tamaño si que es variable ya que no siempre se van a mandar la misma cantidad de datos.
	\item Header y Subheader: son la cabecera y el número del comando recibido por el láser en el anterior mensaje. Sirve de confirmación al usuario de la correcta comprensión del láser a cerca del comando enviado.
	\item Data: como indica su nombre, son los datos de las lecturas realizadas por el láser. Este campo es el único en toda la comunicación el cual no es obligatorio ya que hay casos en los que se puede omitir. Se destacan dos casos:
	\begin{itemize}
		\item Mensaje de error: en los mensajes de error (cuando alguna de las comprobaciones no se ha verificado o la lectura realizada por el láser no se ha terminado con éxito) este campo se omite ya que no hay datos a enviar.
		\item Comandos continuos: en los comandos que provocan un comportamiento de lectura del láser contínua (AR02 o AR04) se envía primero un mensaje con la respuesta que se está describiendo y después se envian mensajes sucesivos (con el mismo comienzo y fin que los otros) donde se muestran los datos.
	\end{itemize}
	\item Status: es el estado del mensaje. A no ser que el mensaje sea un mensaje de error, el resto debería de ser "00" o similar. En caso de que sea un error, estos dos caracteres cambiarán dependiendo del error que ocurra.
	\item CRC: es el código de verificación que demuestra que no ha habido errores en la transmisión, de forma identica a su homólogo en el mensaje enviado en la otra dirección.
\end{itemize}
En el comando empleados en este proyecto (AR02) la forma que posee estos mensajes de respuesta es la siguiente:\\
\imagen{respCorrAR02}{Mensaje de respuesta del láser ante comando AR02}
Como se puede observar, dentro del mensaje que posee los datos de las lecturas se pueden encontrar una grán cantidad de datos que indican otra información relevante sobre la lectura realizada aunque irrelevante para los objetivos de este proyecto, como puede ser el estado de las áreas predeterminadas, los estados de alguno de los puertos de salida, el tiempo que se ha tardado en realizar la lectura, etc.\\
Por todo esto se debe analizar cada trama para extraer la parte en la que exclusivamente aparecen los datos, los cuales después se deben traducir.

\subsection{Traducción de los datos}

Después de haber aislado los datos y haberlos dividido en sus correspondientes partes (ya que cada lectura de entorno es enviada por el láser como una cadena de gran longitud) se debe traducir cada uno de los datos (que posee una longitud de 4 caracteres lo cual se deduce de los 4324 caracteres recibidos divididos entre las 1081 lecturas realizadas por cada lectura de entorno del láser) según una serie de pasos a seguir:
\begin{enumerate}
	\item Se extra el equivalente en hexadecimal de cada uno de los 4 caracteres.
	\item Si el dato hexadecimal está entre los $30_{h}$ y  $39_{h}$, se le resta  $30_{h}$. Por le contrario, si el dato se encuentra entre los  $41_{h}$ y  $46_{h}$, se le resta $37_{h}$.
	\item Se traduce cada dato transformado a codificación binaria.
	\item El comjunto de caracteres binarios transformados en una cadena conjunta se traduce a decimal para extraer el dato expresado en milímetros.
\end{enumerate}
\imagen{decodeData}{Traducción de datos a decimal}
\capitulo{4}{Técnicas y herramientas}

Esta parte de la memoria tiene como objetivo presentar las técnicas metodológicas y las herramientas de desarrollo que se han utilizado para llevar a cabo el proyecto. Si se han estudiado diferentes alternativas de metodologías, herramientas, bibliotecas se puede hacer un resumen de los aspectos más destacados de cada alternativa, incluyendo comparativas entre las distintas opciones y una justificación de las elecciones realizadas. 
No se pretende que este apartado se convierta en un capítulo de un libro dedicado a cada una de las alternativas, sino comentar los aspectos más destacados de cada opción, con un repaso somero a los fundamentos esenciales y referencias bibliográficas para que el lector pueda ampliar su conocimiento sobre el tema.


\subsection{Desarrollo de la memoria}

En este aspecto se ha querido emplear LaTeX en contraposición a otros programas de propósito similar como pueden ser Microsoft Word o Open Office Writer ya que esta aplicación nos permite una mayor cantidad de posible acciones a realizar en el documento a realizar. \hfill
\break\break
Por ejemplo, a presentación del documento puede quedar más limpia con esta aplicación ya que el usuario puede manejar el espacio de cada hoja a su antojo y poder disponer de todo él ya que se puede variar los márgenes de las páginas de una forma sencilla (introducción de un comando), lo que en los otros programas antes mencionados resultaría demasiado complicado a demás de ser potencialmente peligroso debido a que, gracias a las acciones automáticas que poseen, podrían desajustar todo el contenido de la hoja en sí haciendo que el usuario tenga que ajustar cada uno de los elementos que antes poseía. LaTeX permite saltarse los márgenes establecidos, por ejemplo, para insertar una imagen. en cuanto a los encabezados, pies de página y numeración de página también se puede realizar con un simple comando. \hfill
\break\break
Con todo esto y con la característica de que permite guardar el archivo directamente en PDF sin realizar ninguna conversión (aunque para ello necesite crear algún archivo debido a la compilación) es por la que se ha escogido para este 

\subsection{Desarrollo del código}

Debido a que es un programa familiar debido a su utilización en diferentes asignaturas a lo largo del grado, se ha escogido Eclipse para el desarrollo de esta parte del proyecto. Para poder emplearlo de forma dirigida a este proyecto se ha necesitado integrar este programa con el entorno de desarrollo conocido como MTX-GTW. \hfill
\break\break
Este entorno es el que permitirá la extracción de medidas necesarias para saber la distancia a la que se encuentra un objeto del láser si es que se detecta un objeto. Otra característica de este entorno es la compatibilidad con el sistema operativo, ya que solo es compatible con sistemas Ubuntu y demás distribuciones de Linux. \hfill
\break\break
Como ya se ha dicho en la introducción,  para el desarrollo del código se necesita tener la máquina virtual del sistema operativo Ubuntu 16.04. LTS. Este sistema se ha escogido, al igual que Eclipse, debido a que se han empleado en otras asignaturas del grado con lo que resulta familiar para poder trabajar de manera más cómoda y eficiente que si se debiera aprender su modo de empleo. A demás de estas características, se ha escogido por el tema de la compatibilidad explicado anteriormente.\hfill

\subsection{Planteamiento de las tareas}

Para organizar las actividades a realizar se ha escogido la aplicación Trello. Esta aplicación se ha escogido debido a su facilidad de uso (debido a que su interfaz de usuario es muy intuitiva), a demás de otros muchos aspectos como el hecho de que a un tablero (unidades organizativas en las que se gestionan las tareas de los diferentes proyectos que se puede gestionar desde una misma cuenta de usuario) se puede acceder más de un usuario. Esta característica permite que tanto el alumno como el profesor pueden acceder al mismo tablero para gestionar las tareas a realizar.

\subsection{Metodología de gestión y herramientas asociadas}

Para este proyecto se ha decidido utilizar la metodología de SCRUM. Esta metodología esta basada en entregas incrementales pero funcionales. Para la realización de esta metodología, perteneciente a las denominadas metodologías ágiles, se va a utilizar la aplicación ZenHub. Esta aplicación se puede emplear para presentar los sprints y se puede planificar la fecha de comienzo y final cada una de las tareas.

\subsection{Patrones de diseño empleados}

Los patrones de diseño son herramientas reutilizables empleadas para resolver problemas que resultan comunes a la hora tanto de desarrollar el software como el diseño de las interacciones e interfaces empleadas para que el usuario pueda emplear el sistema.\hfill
\break\break
En el desarrollo de el sistema en el que se basa el proyecto se han utilizado los siguientes patrones:
\begin{itemize}
    \item Plantilla: para la realización de las clases \textit{Láser} e \textit{Interfaz} ya que aunque cada interfaz y cada láser tienen ciertas funciones específicas hay una parte de estas que son comunes por lo que, para evitar tener que duplicar código y mejorar la eficiencia de estas partes del sistema.\hfill
\break
    \item Singleton: en este proyecto se va a crear la clase \textit{Procesador} como un Singleton ya que solo se debería de crear una instancia de este tipo ya solo es necesario esta instancia para el funcionamiento del  sistema. Con la utilización de este patrón nos aseguramos de solo tener una instancia de la clase antes mencionada lo que mejora el control del flujo de datos a demás de la eficiencia del sistema.\hfill
    \break
\end{itemize}
\capitulo{5}{Aspectos relevantes del desarrollo del proyecto}

\section{Definición del trabajo como proyecto de investigación}

Un aspecto que hay que destacar de este proyecto es que su enfoque, además de ser el de crear un sistema funcional que cumpla los requisitos establecidos en puntos anteriores, es el de crear una simulación del entorno en el que posteriormente se implementará este sistema, buscando los elementos tanto hardware como software (programas, lenguajes, librerías, recursos...)  necesarios para que un sistema ya implantado en su entrono de trabajo pueda utilizar el sistema desarrollado para realizar algunas de sus tareas. Como resumen, se podría decir que es un trabajo que pretende desarrollar un sistema para la investigación de la implementación de un nuevo sistema de detección de objetos.\\

\section{Conocimientos utilizados aprendidos en el grado}

Algunos de los conocimientos que se han empleado en este proyecto se han adquirido a lo largo de las diferentes asignaturas de los 4 cursos del grado.\\

\subsection{Codificación en Python}
El conocimiento principal que ha permitido desarrollar el código funcional de este proyecto es el conocimiento a cerca de la creación de código en lenguaje Python. Gracias a algunas asignaturas como son Sistemas Inteligentes ($1^{er}$ semestre del $3^{er}$ curso), Algoritmia (2º semestre del $3^{er}$ curso), Nuevas Tecnologías y empresa y Minería de Datos (ambas del 2º semestre del 4º curso).\\
\\
Gracias a haber recibido clases de estas asignaturas se han podido conocer y aprender a manejar algunas de las librerías que se han empleado en la creación de este sistema software, como pueden ser Numpy, Matplotlib o Math, además de los conocimientos a cerca de las características principales de este lenguaje de programación.\\

\subsection{Conocimiento sobre sockets}
Aunque en este sistema se ha tratado este asunto con poco detalle, el manejo de sockets también es un concepto que se ha sabido emplear gracias a alguna asignatura del grado. Un ejemplo de estas asignaturas es Sistemas Distribuidos (2º semestre del 4º curso).\\
\\
Gracias a los conocimientos adquiridos en esta asignatura se ha podido configurar y gestionar el comportamiento del socket lógico creado para la comunicación entre el láser y el ordenador.\\

\subsection{Conocimiento a cerca de la gestión del proyecto}
Este aspecto es uno de los más relevantes ya que es la parte del proyecto la cual ayuda a gestionar el tiempo que se usa para llegar al objetivo marcado por los requisitos del proyecto. Los conceptos de la definición, aspectos y formas de aplicación de las metodologías ágiles, SCRUM en el caso de este proyecto, han sido adquiridos gracias a haber recibido las clases de la asignatura de Gestión de Proyectos ($1^{er}$ semestre del $3^{er}$ curso).\\

\section{Conocimientos externos a lo aprendido en el grado}

También existen conocimientos aprendidos gracias al desarrollo de este proyecto los cuales no se han adquirido a lo largo del grado.\\
En cuanto a este proyecto, se ha de destacar uno solo de estos conocimientos, la creación de los mensajes. Como ya se ha explicado en apartados anteriores, para comunicarse con el láser, el PC debe enviar una serie muy concreta de comandos, los cuales deben de temen una codificación específica. Esta codificación se basa en que algunas de las partes del mensaje (inici y final del mensaje y CRC) se deben enviar codificadas como Bytes mientras que el resto deben ser enviados como código ASCII. Ninguna de estas codificaciones ha sido impartida en ninguna de las asignaturas que imparten código en Python mencionadas en el punto anterior.\\
\\
Estos conocimientos tienen la ventaja de que, al ser conceptos sobre uno de los  lenguajes de programación más usados, es sencillo encontrar mucha información al respecto por internet, por lo que se pueden encontrar diferentes formas de obtener el mismo  resultado (utilizando diferentes librerias o creando métodos con las librerías que están por defecto en el propio lenguaje) con lo que el programador puede escoger la que le parezca más idónea para su propósito.
\capitulo{6}{Trabajos relacionados}

En este apartado hablaremos de proyectos similares al realizado,  que existen actualmente en el mercado.

\section{Grafana}
Es el trabajo relacionado más famoso de todos, y es empleado para monitorizar todo tipo de datos. 

A diferencia de mi proyecto, grafana permite cargar directamente los datos desde una base de datos, lo que es muy útil en empresas que quieran mostrar datos a sus clientes.

Grafana es gratuita (open source)\cite{GRAFANA}, y está basada en cuadros de mandos fácilmente editables , que hacen de ésta una herramienta muy potente, donde el usuario selecciona que datos ver, y como los quiere ver; dentro de las diversas opciones que te permiten seleccionar.

A pesar de ello, los gráficos que ofrece a veces son bastante confusos, y no esta planteado para trabajar con cursores, a pesar de ser un proyecto que admite muchas funcionalidades adicionales, al ser código open source.

\section{Graphite}
Al igual que grafana, graphite es una herramienta open source que permite monitorizar y graficar datos de sistemas informáticos en tiempo real.
Destaca por la exactitud del tiempo real con la que ofrece sus datos.

Graphite funciona de la siguiente forma\cite{GRAPHITE}:
\begin{itemize}
	\item Un demonio en segundo plano, escucha los datos a tiempo real.
	\item Estos datos se guardan en una base datos.
	\item Una webapp llamado Django, renderiza los puntos en tiempo real.
\end{itemize} 

\section{ChartBlocks}

Este es un editor muy fácil de usar, que permite realizar gráficos con relativa facilidad.

A diferencia del mi proyecto, y de los dos anteriores, destaca por su facilidad de uso, aunque en uso, no es tan potente como los anteriores.\cite{CHARTBLOCKS}

Permite leer datos de distintas fuentes, incluso de datos en tiempo real y destaca la facilidad de su interfaz.




\capitulo{7}{Conclusiones y Líneas de trabajo futuras}

Todo proyecto debe incluir las conclusiones que se derivan de su desarrollo. Éstas pueden ser de diferente índole, dependiendo de la tipología del proyecto, pero normalmente van a estar presentes un conjunto de conclusiones relacionadas con los resultados del proyecto y un conjunto de conclusiones técnicas. 
Además, resulta muy útil realizar un informe crítico indicando cómo se puede mejorar el proyecto, o cómo se puede continuar trabajando en la línea del proyecto realizado. 
\section{Introducción}
En este apartado se expondrán las conclusiones derivadas del desarrollo del proyecto y de los resultados finales obtenidos. Por otra parte, se expondrán una serie de posibles mejoras y líneas de trabajo futuras con respecto a este proyecto.

\section{Conclusiones}
Finalizando la creación de esta memoria, se expondrán las conclusiones obtenidas del desarrollo de este proyecto:
\begin{itemize}
	\item Como primera de las conclusiones se ha de destacar la parte del proyecto que más ha costado conseguir o más bien la que más ha costado de conseguir de forma estable,el entorno de programación. Esta parte es una de las que más tiempo ha llevado si no es la que más debido a los problemas surgidos debido a programas como Eclipse, el cual fallo impidiendo continuar el proyecto durante un periodo de tiempo muy extenso, al igual que el uso de sistemas virtualizados como es el caso del uso del sistema operativo Ubuntu en el programa Oracle VirtualBox que al no encontrar solución posible para el problema encontrado se tuvo que rehacer varias veces. Aunque haya supuesto un problema, ha servido también para conocer estas aplicaciones en profundidad y poder investigar para solucionar estos fallos o similares en un futuro.
	\item Otra conclusión a tener en cuenta es la cantidad de nuevos conocimientos obtenida gracias al desarrollo de este software. Se ha obtenido nueva información a cerca de comunicación entre dispositivos, sockets, formas de codificación de diferentes conjuntos de datos, diferentes formas de implementación de un sistema para obtener los mismos resultados (ya sea variando los elementos hardware o software utilizados), las ventajas y desventajas que supone crear una determinada parte del sistema dependiendo del lenguaje de programación utilizado para ello, etc.
	\item Y hablando de las investigaciones, este proyecto ha servido también para conocer como realizar un proyecto software basado en investigaciones a cerca de temas de los que se tenía un idea muy superficial pero se ha podido descubrir más aspectos los cuales han servido para aumentar conocimientos y hacer saber la dinámica llevada a cabo en proyectos de investigación. 
\end{itemize}
En resumen, la experiencia en el desarrollo de este proyecto ha sido por lo general satisfactoria ya que, aún siendo una actividad realmente costosa en algunos aspectos, ha sido útil para dar a conocer la realidad sobre el proceso que este tipo de proyectos siguen para llevarse a cabo y conocer también las facilidades y dificultades que estos suponen para las personas encargadas de su desarrollo.

\section{Mejoras y líneas de trabajo}

\subsection{Funcionamiento continuo}
En el sistema desarrollado, al ejecutarlo, solos se trabaja con una lectura de entorno por ejecución. Esto es debido a que, aún utilizando el comando de lectura continua AR02 no se plantea la lectura continua sino que se reciben los datos correspondientes a la primera de las lecturas antes de cerrar el socket qe conecta los elementos del sistema. Esto es debido a la velocidad por la cual el sistema analiza los datos que recibe.\\
\\
Al no ser un análisis inmediato, si se intentasen analizar todas las lecturas recibidas del láser el sistema tendría un retardo demasiado grande para que sea útil para su uso a nivel industrial. Además este retardo sería exponencial ya que, mientras se analiza una de las tramas de datos, el sistema ya ha recibido varias más ya que el láser realiza y envía una trama de datos (correspondiente a una lectura del entorno) cada 30 milisegundos (ms).\\
\\
Para este aspecto se pueden plantear varias líneas de trabajo:
\begin{itemize}
	\item Mejora en la eficiencia: se puede modificar el código para hacerlo más eficiente, lo que disminuiría el retraso con respecto a los datos recibidos y los analizados.
	\item Cambio de mensaje: también es posible el cambio en cuanto a la comunicación con el láser ya que se puede enviar el mensaje AR01 en lugar del AR02 con lo que se recibiría una única lectura por cada paso de mensaje, evitando la acumulación de datos.
	\item Programación concurrente: otro aspecto a analizar es la programación multihilo de este software, ya que se puede crear diferentes hilos que puedan ir recibiendo datos, analizando datos y representando gráficamente dichos datos respectivamente, lo cual también evitaría o por lo menos reduciría el retardo antes mencionado.
\end{itemize}

\subsection{Ejecución remota}
Esta mejora viene motivada por, como ya se ha explicado en otros apartados, el hecho de no haber conseguido implementarlo durante el desarrollo de este proyecto.\\
\\
Para el desarrollo de este aspecto se necesita la inclusión en el sistema del MTX-GTW. Este instrumento es capaz de simular los recursos, y por tanto el rendimiento, presentes en un AGV. Para comunicarse con él se debe usar varios cables adicionales:
\begin{itemize}
	\item Cable de red: para realizar la conexión entre el MTX-GTW y el ordenador se necesita un cable de red estándar.
	\item Cable con conector RS232: este cable es necesario para poder conectar el nuevo dispositivo con el láser.
	\item Cable de alimentación: se necesitará tambien un cable que suministre 12 V de conrriente continua sin cabezal ya que posee uno propio al cual se han de acoplar dos hilos individuales.
\end{itemize}
Tras haber obtenido este sistema hardware se ha de comprobar si en el MTX-GTW se puede ejecutar código en Python, ya que si no es así sería necesario traducir el sistema a lenguajes como C$++$, el cual si que es capaz de comprender.

\bibliographystyle{plain}
\bibliography{bibliografia}{}


\end{document}
